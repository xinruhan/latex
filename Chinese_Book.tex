%!TEX program = xelatex

\documentclass{progbookcn}
\usepackage{graphicx}
\usepackage[dvipsnames]{xcolor}
\usepackage{wrapfig}
\usepackage{enumerate}
\usepackage{amsmath,mathrsfs,amsfonts}
\usepackage{tabularx}
\usepackage{booktabs}
\usepackage{colortbl}
\usepackage{multirow,makecell}
\usepackage{multicol}
\usepackage{ulem} % \uline
\usepackage{listings}
\usepackage{tikz}
\usepackage{tcolorbox}
\usepackage{fontawesome5}
\usepackage{lipsum}
\usepackage{threeparttable}

%%%%%%%%%%%%%%%%%%%%%%%%%%%%%%%%%%%%%%%%
%% beginning of the book
%% 文档开始
%%%%%%%%%%%%%%%%%%%%%%%%%%%%%%%%%%%%%%%%

\begin{document}

%% title page
\begin{titlepage}
  \vspace*{25ex}

  \hspace{0.05\textwidth}\begin{minipage}{.9\textwidth}
    \flushright

    %%中文书名
    {\zihao{1}\textbf{中级计量经济学讲义}}

    \rule{\linewidth}{.5pt}

    \vspace{2ex}

    %% 英文书名
    {\zihao{3}\textbf{Notes of Intermediate Econometrics}} \\

    \vspace{20ex}

    %% 作者
    {\zihao{4}\heiti{韩昕儒}} \\
    {\zihao{5}中国农业科学院农业经济与发展研究所}
  \end{minipage}

  \vfill

  \centering
  {\zihao{4}\number\year 年 \number\month 月 \number\day 日初稿}
\end{titlepage}
\thispagestyle{empty}


\frontmatter


\chapter{序}

待更新。

\chapter{前言}

待更新。

%% 目录
\clearpage
{
  \hypersetup{hidelinks}
  \tableofcontents
}


\mainmatter


\part{宏观形势}


\chapter{宏观经济:回顾与展望}
\begin{titledbox}{\pfhei{主要观点}}
\begin{itemize}
  \item 2019年世界经济增速同步放缓,通货膨胀率提升;预计2020年世界经济温和回升,发达经济体的增长势头趋缓,新兴经济体的增长动能有望增强。
  \item 2019年中国经济运行总体平稳,居民消费价格预期上涨,城乡居民人均收入比值继续缩小,经济转型升级态势持续,对外贸易逆势增长,就业形势保持稳定;预计2020年中国经济增速缓中趋稳,投资增速小幅回升,外需不确定性出现缓和迹象。
 \end{itemize}
\end{titledbox}

\section{2019年世界宏观经济走势与2020年展望}
\subsection{2019年世界宏观经济走势}
\textbf{一、世界经济增速同步放缓,下行风险增大}

受贸易摩擦、地缘政治和衰退风险等重大不确定因素的共同影响,2019年的世界经济面临着不断下行的压力。发达国家和新兴经济体的经济增速出现了同步放缓。世界银行预计2019年全球增速为2.9\%,同比下降0.7\%(见表1-1),为2008年金融危机以来的最低水平。

在主要发达经济体中,只有日本经济增速表现出缓慢上升趋势,美国及欧元区等经济体增速回落。2019年,发达经济体GDP增速为1.7\%,同比降低0.5\%。其中,美国GDP增长2.3\%,同比降低0.6\%;欧元区GDP增长1.2\%,同比降低0.7\%;日本GDP增长1.0\%,同比提高0.7\%。

新兴市场与发展中经济体发展速度与2018年相比整体下降。2018年,新兴市场与发展中经济体GDP增速3.7\%,同比下降0.8\%。2019年印度GDP增长4.8\%,同比降低2.0\%,欧洲新兴和发展中经济体GDP增速1.8\%,同比下降1.3\%,墨西哥GDP增长一度降为0\%,同比下降2.1\%。

\textbf{二、全球通货膨胀率温和上涨}

2019年年初至今全球通胀水平因经济增长疲弱和能源价格回落保持温和水平。由于工资增长疲弱,撇除食品及能源价格外的核心通胀,整体上亦维持温和态势;2019年9月,美国、日本和欧元区三大经济体CPI同比涨幅分别为2.3\%、0.8\%、1.4\%,美国通货膨胀率与上年同期相比并无变化,日本、欧元区与上年同期相比下降了0.4个和0.7个百分点。一直以来备受通货膨胀困扰的巴西和俄罗斯,2019年9月,巴西和俄罗斯CPI分别同比增长4.19\%和2.4\%。


\begin{table}[]
\centering
\begin{threeparttable}
\textbf{\caption{2018-2021年世界经济增长概况}}
\small
\setlength{\tabcolsep}{9mm}
\begin{tabular}{@{}lcccc@{}}
\hline
国家与地区       & 2018年 & 2019年 & 2020年(预计) & 2021年(预计) \\ 
\hline
世界经济        & 3.6   & 2.9   & 3.3       & 3.4       \\
发达经济体       & 2.2   & 1.7   & 1.6       & 1.6       \\
美国          & 2.9   & 2.3   & 2.0         & 1.7       \\
欧元区国家       & 1.9   & 1.2   & 1.3       & 1.4       \\
日本          & 0.3   & 1.0     & 0.7       & 0.5       \\
英国          & 1.3   & 1.3   & 1.4       & 1.5       \\
新兴市场和发展中经济体 & 4.5   & 3.7   & 4.4       & 4.6       \\
亚洲新兴和发展中经济体 & 6.4   & 5.6   & 5.8       & 5.9       \\
中国          & 6.6   & 6.1   & 6.0         & 5.8       \\
印度          & 6.8   & 4.8   & 5.8       & 6.5       \\
欧洲新兴和发展中经济体 & 3.1   & 1.8   & 2.6       & 2.5       \\
俄罗斯         & 2.3   & 1.1   & 2.9       & 2.0         \\
拉丁美洲和加勒比地区  & 1.1   & 0.1   & 1.6       & 2.3       \\
巴西          & 1.3   & 1.2   & 2.2       & 2.3       \\
墨西哥         & 2.1   & 0.0     & 1.0        & 1.6       \\
中东和东亚       & 1.9   & 0.8   & 2.8       & 3.2       \\
沙特阿拉伯       & 2.4   & 0.2   & 1.9       & 2.2       \\
撒哈拉以南非洲     & 3.2   & 3.3   & 3.5       & 3.5       \\
尼日利亚        & 1.9   & 2.3   & 2.5       & 2.5       \\
南非          & 0.8   & 0.4   & 0.8       & 1.0         \\ 
\hline
\end{tabular}
\begin{tablenotes}
\item \tiny{资料来源:世界银行2020年1月发布的《全球经济展望》。}
\end{tablenotes}
\end{threeparttable}
\end{table}


\subsection{2020年世界宏观经济展望}
\textbf{一、世界经济温和回升,下行风险减弱但依旧突出}

世界银行对2020年全球经济增速预期为3.3\%,较2019年相比上升0.4\%,并且2019年发达经济体和新兴市场经济体大规模货币宽松的效果预计将在2020年继续对全球经济产生影响,全球复苏预计将伴随着贸易增速回升,尤其反映出国内需求和投资的复苏,以及汽车和技术部门一些临时性拖累因素的消退,世界经济温和上升。其次全球增长前景面临的风险依然偏向下行,但下行风险弱化,主要由于中美达成“第一阶段”贸易协议,英国避免“无协议退出欧盟”等,然而下行风险依然突出,地缘政治紧张局势(尤其是美国和伊朗之间)加剧冲击全球复苏;美国与其贸易伙伴(尤其是中国)之间贸易壁垒的抬高挫伤了商业情绪,加剧了去年很多经济体经历的周期性和结构性放缓;近年来,与气候相关的灾难,如热带风暴、洪水、高温天气、干旱和野火已经在多个地区造成了严重的人道主义和人类生计损失。气候变化——气候相关灾难频率提升、强度增大的驱动因素——已经危及到健康和经济结果,这不局限于直接受到影响的区域,很可能给其他还未感受到直接影响的地区带来挑战。

\textbf{二、主要经济体经济增长将进一步分化}

主要经济体增长受到的影响因素存在差异,发达经济体生产率增长乏力;新兴市场和发展中经济体增长分化较为明显。根据世界银行预测预计,2020年美国的经济增速将由2019年的2.3\%回落到2.0\%;欧元区国家经济增速将由2019年的1.2\%上升至1.3\%;英国于2020年1月底从欧盟有序退出,而后逐步过渡至一种新的经济关系,预计2020年经济增速稳定在1.4\%;日本的经济增速预计将从2019年估计的1\%降至2020年的0.7\%;新兴市场和发展中经济体正从深度下滑中恢复,新兴和发展中亚洲的经济增速预计将从2019年的5.6\%小幅增长至2020年的5.8\%;新兴和发展中欧洲的经济增速预计将从2019年的1.8\%上升至2020—2021年的2.5\%左右;2020年中东和东亚地区经济增长2.8\%,较2019年提升了2个百分点;拉丁美洲和加勒比地区2020年经济增速由0.1\%提升到1.6\%;撒哈拉以南非洲2020年经济增速由2019年的3.3\%至2020—2021 年的3.5\%。

\section{2019年中国宏观经济走势与2020年展望}
\subsection{2019年中国宏观经济走势}
\textbf{一、国民经济运行总体平稳}

2019年,国民经济运行总体平稳,发展质量稳步提升,主要预期目标较好实现,为全面建成小康社会奠定了坚实基础。初步核算,全年国内生产总值990865亿元,按可比价格计算,比上年增长6.1\%,符合6\%~6.5\%的预期目标。分季度看,中国GDP在2019年第一季度同比增长6.4\%,第二季度同比增长6.2\%,第三季度同比增长6.0\%,第四季度同比增长6.0\%。分产业看,第一产业增加值70467亿元,比上年增长3.1\%;第二产业增加值386165亿元,同比增长5.7\%;第三产业增加值534233亿元,同比增长6.9\%。

\textbf{二、居民消费价格涨幅符合预期目标,工业生产者价格小幅下降}

全年居民消费价格比上年上涨2.9\%,符合3\%左右的预期目标。其中,城市上涨2.8\%,农村上涨3.2\%。分类别看,食品烟酒价格上涨7.0\%,衣着上涨1.6\%,居住上涨1.4\%,生活用品及服务上涨0.9\%,交通和通信下降1.7\%,教育文化和娱乐上涨2.2\%,医疗保健上涨2.4\%,其他用品和服务上涨3.4\%。在食品烟酒价格中,粮食价格上涨0.5\%,鲜菜价格上涨4.1\%,猪肉价格上涨42.5\%。扣除食品和能源价格的核心CPI上涨1.6\%,涨幅比上年回落0.3个百分点。12月份,居民消费价格同比上涨4.5\%,环比与上月持平。全年工业生产者出厂价格比上年下降0.3\%,12月份同比下降0.5\%,环比与上月持平。全年工业生产者购进价格比上年下降0.7\%,12月份同比下降1.3\%,环比与上月持平。

\textbf{三、居民收入增长与经济增长基本同步,城乡居民人均收入比值继续缩小}

全年全国居民人均可支配收入30733元,比上年名义增长8.9\%,增速比上年加快0.2个百分点;扣除价格因素实际增长5.8\%,与经济增长基本同步,与人均GDP增长大体持平。按常住地分,城镇居民人均可支配收入42359元,比上年名义增长7.9\%,扣除价格因素实际增长5.0\%;农村居民人均可支配收入16021元,比上年名义增长9.6\%,扣除价格因素实际增长6.2\%。城乡居民人均收入比值为2.64,比上年缩小0.05。全国居民人均消费支出21559元,比上年名义增长8.6\%,增速比上年加快0.2个百分点;扣除价格因素实际增长5.5\%。按常住地分,城镇居民人均消费支出28063元,名义增长7.5\%;农村居民人均消费支出13328元,名义增长9.9\%。

\textbf{四、对外贸易逆势增长,一般贸易占比继续提升}

全年货物进出口总额315446亿元,比上年增长3.4\%。其中,出口172298亿元,增长5.0\%;进口143148亿元,增长1.6\%。进出口相抵,顺差为29150亿元。一般贸易进出口占进出口总额的比重为59.0\%,比上年提高1.2个百分点。机电产品出口增长4.4\%,占出口总额的58.4\%。我国对欧盟、东盟进出口分别增长8.0\%和14.1\%;与“一带一路”沿线国家进出口增势良好,对“一带一路”沿线国家合计进出口增长10.8\%,高出货物进出口总额增速7.4个百分点。全年规模以上工业企业实现出口交货值124216亿元,比上年增长1.3\%。

\textbf{五、重点改革和攻坚任务扎实推进,经济转型升级态势持续}

供给侧结构性改革成效显著,经济结构继续优化。全年第三产业增加值占国内生产总值的比重为53.9\%,比上年提高0.6个百分点,高于第二产业14.9个百分点;对国内生产总值增长的贡献率为59.4\%。消费作为经济增长主动力作用进一步巩固,最终消费支出对国内生产总值增长的贡献率为57.8\%,高于资本形成总额26.6个百分点。居民消费升级提质。全国居民恩格尔系数为28.2\%,比上年下降0.2个百分点。全年全国居民人均消费支出中,服务性消费支出占比为45.9\%,比上年提高1.7个百分点。

\textbf{六、就业形势保持稳定,城镇调查失业率符合预期目标}

全年城镇新增就业1352万人,连续7年保持在1300万人以上,明显高于1100万人以上的预期目标,完成全年目标的122.9\%。12月份,全国城镇调查失业率为5.2\%,2019年各月全国城镇调查失业率保持在5.0\%~5.3\%之间,实现了低于5.5\%左右的预期目标。全国主要就业人员群体25-59岁人口调查失业率为4.7\%。12月份,31个大城市城镇调查失业率为5.2\%。2019年末,城镇登记失业率为3.62\%,比上年末降低0.18个百分点,符合4.5\%以内的预期目标。年末全国就业人员77471万人,其中城镇就业人员44247万人。全年农民工总量29077万人,比上年增加241万人,增长0.8\%。其中,本地农民工11652万人,增长0.7\%;外出农民工17425万人,增长0.9\%。农民工月均收入水平3962元,比上年增长6.5\%。

\subsection{2020年中国宏观经济走势展望}

\textbf{一、坚持稳中求进 保持经济运行在合理区间}

中国经济增速下滑但不悲观,2020年中国经济增速仍将继续回落,按照目前的测算,经济增速只要维持5.5\%左右,就业就能稳住。2020年中国经济增长的区间管理目标宜设为5.5\%~6\%,保守目标为5.8\%左右,不仅能够完成“两个一百年”目标的阶段性任务和保证社会就业的基本稳定,也更有利于保持战略定力,按照既定方针推动经济高质量发展。预计2020年出口增速可能回升到正增长,上半年增速较低,下半年有所回升。需求状况未见显著改善,但是鼓励进口力度加大,加大对美国农产品进口。

\textbf{二、投资增速小幅回升,外需不确定性出现缓和迹象}

2020年投资增速小幅回升,不同类别投资增速升降互现。2020年稳投资的关键在于稳基建和稳制造业,房地产开发投资增速可能逐渐下行,整体投资增速可能小幅回升到5.6\%左右。此外随着中美经贸摩擦似有缓和迹象,外需有望不再在2020年对增长形成拖累。

\chapter{农业产业:回顾与展望}
\begin{titledbox}{\pfhei{主要观点}}
\begin{itemize}
  \item 2019年,世界农产品产量整体增速放缓,非洲猪瘟对中国等东亚国家的猪肉生产影响比预期更为严重,农产品需求总体趋于稳定,全球贸易紧张局势加剧导致农产品贸易增长显著放缓。预计2020年,全球谷物和肉类产量保持上升趋势,人均占有量稳步提高,新冠肺炎疫情和沙漠蝗灾情加剧国际农产品价格震荡,给农产品贸易带来极大的不确定性。
  \item 2019年,中国农业发展稳中有进、稳中向好,农产品生产保持稳定,市场运行基本平稳,农产品贸易继续增长;农业供给侧结构性改革初见成效,农业高质量发展进一步提升,农业结构进一步优化,农业绿色发展持续推进,农业科技进步贡献率达到59.2\%;农村一二三产业融合水平不断提高,新型农业经营主体稳步发展。预计2020年,中国粮食作物产量持平略增,进口压力加大,预计粮食产量将达到6.7亿吨;经济作物供需基本保持稳定,品种结构有所调整;畜牧水产:供需相对处于紧平衡,猪肉生产仍存在较大不确定性,预计产量较2019年降低10\%。
 \end{itemize}
\end{titledbox}

\section{2019年世界农业产业走势与2020年展望}
\subsection{2019年世界农业产业走势回顾}
\textbf{一、农产品生产:农产品产量整体增速放缓}

2019年全球农产品产量整体增速放缓。根据联合国联合国粮食及农业组织(FAO)预估\footnote{若无特殊说明,本部分数据来自于联合国联合国粮食及农业组织(FAO)于2019年11月发布的《粮食展望》(Food Outlook)。} ,2019年,全球小麦产量约为7.65亿吨,达到新的历史高点。粗粮产量将从2018年低迷中复苏,达到14.25亿吨,这主要得益于大麦产量的增长(1340万吨)。全球水稻产量将比上一年的历史最高水平降低0.8\%,仍然保持高位运行。全球油籽产量也较上年度有所下降,主要反映在大豆和油菜籽产量的降低,其中,大豆产量减少主要是由于美国大豆种植面积缩小;而油菜籽方面,不确定的出口前景限制了加拿大的油菜籽种植,同时,欧盟和澳大利亚等油菜籽主产国的旱干天气严重影响了今年收成。此外,棕榈油全球产量也有所放缓,这与印尼和马来西亚棕榈油产量前景不佳有关。全球食糖产量目前预计略高于1.75亿吨,较2018年度下降2.8\%。在畜牧生产方面,2019年,全球肉类产量约为3.35亿吨,比2018年低1.0\%,主要原因是非洲猪瘟(ASF)对中国等东亚国家的猪肉生产影响比预期更为严重,中国肉类产量预计将下降12\%。全球鱼类产量将与上年同期持平,捕捞渔业产量下降3.4\%,与水产业产量增加3.9\%相抵消。

\begin{table}[]
\centering
\begin{threeparttable}
\textbf{\caption{2019年世界谷物生产、消费、库存和贸易}}
\small
\setlength{\tabcolsep}{6.25mm}
\begin{tabular}{@{}lcccc@{}}
\hline
       & 小麦 & 水稻 & 粗粮 & 谷物合计 \\ 
\hline
生产量(百万吨)& 765(4.5\%)&   531.4(-0.8\%)&  1425.5(1.2\%)&  2721.9 \\ 
贸易量(百万吨) & 172.1(2.3\%)&   47.7(3.1\%)&  195.3(-1.2\%)&  415.1 \\
消费量 (百万吨)& 759.5(1.5\%)&   515.9(1.1\%)&   1433.9(0.2\%) &  2709.3 \\
~~~~食用  (百万吨)& 517.8(1.1\%) &  417.8(1.7\%) &  216.4(0.1\%)&   1152 \\
~~~~饲料  (百万吨)& 144.4(2.8\%) &  / &  805(-0.2\%)&  949.4 \\
~~~~其他用途(百万吨) &  97.2(-2.3\%)&   / &  412.5(1.1\%)  &  509.7 \\
期末库存(百万吨) &  274.9(1.9\%)&   180.9(-1.2\%) &  393.6(-3.9\%) & 849.4 \\
FAO 价格指数
(2002—2004年=100) &  143(-3.1\%)&  224(-1.0\%)&  162(3.7\%)&   / \\
\hline
\end{tabular}
\begin{tablenotes}
\item \tiny{资料来源:联合国粮食及农业组织(FAO)2019年11月发布的《粮食展望》。}
\item \tiny{注:括号内为2019年的同比增长率;“/”代表缺少数据。}
\end{tablenotes}
\end{threeparttable}
\end{table}

\begin{table}[]
\centering
\begin{threeparttable}
\textbf{\caption{2019年世界糖类生产、消费、库存和贸易}}
\small
\setlength{\tabcolsep}{10.5mm}
\begin{tabular}{@{}lcccc@{}}
\hline
       &2017年 & 2018年 & 2019年 & 2019年同比增长率 \\ 
\hline
生产量(百万吨) &183.2 &180.1 &1175.1 & -2.8\% \\
贸易量(百万吨) &61.7  &55.6  &56.3  &1.3\% \\
消费量(百万吨) &172.3 &175.0 &177.5 &1.4\% \\
期末库存(百万吨)&  89.0&  93.9&  91.4&  -2.6\% \\
\hline
\end{tabular}
\begin{tablenotes}
\item \tiny{资料来源:联合国粮食及农业组织(FAO)2019年11月发布的《粮食展望》。}
\end{tablenotes}
\end{threeparttable}
\end{table}
\textbf{二、农产品需求:世界经济下行压力较大,需求趋于稳定}

据FAO最新预测,2019年,全球小麦总消费量将达到7.595亿吨,比2018年高出1.5\%,其中,小麦的粮食总消费量预计将接近5.18亿吨,增长1.1\%。粗粮总消费量预计将保持接近于2018年的水平,大麦消费量增长强劲(较2018年增长了近5\%)。2019年,玉米的饲料消费量出现了近10年来的首次萎缩,这主要是因为美国的玉米饲料消费量从2018年度的创纪录高位大幅下降,同时,玉米饲料在几个亚洲国家的消费量受到非洲猪瘟生猪产量减少的影响。2019年,全球食糖消费量将增长1.4\%,低于长期(10年)消费量增长趋势,反映出消费者对糖消费过剩的担忧日益加剧。在肉制品方面,2019年初以来连续出现国际肉类价格环比上涨,其中猪肉(尤其是冷冻猪肉)价格涨幅最大,主要由于中国进口需求激增。而受到亚洲需求走强的支撑,家禽、绵羊和牛肉价格也持续走高。不同的是,与去年同期相比,由于供应增加和需求下降的结果FAO鱼类价格指数在1月至9月期间下降了2.1\%。

\textbf{三、农产品贸易:全球贸易紧张局势加剧,农产品贸易增长显著放缓}

2019年世界经济持续下行,贸易紧张局势加剧,全球农产品贸易增长显著放缓。2019年,全球小麦贸易预估约为1.72亿吨,俄罗斯联邦仍然是世界最大的小麦出口国。虽然北非和沙特阿拉伯对大麦的进口需求增加,使全球大麦贸易较2018/年度增长近9\%,但受玉米贸易下降的影响,2019年全球粗粮贸易将较2018年仍有所下降。全球食用油和食糖贸易都有小幅度增加。全球肉类和肉制品贸易预计将达到3600万吨,比2018年增长6.7\%,中国为解决国内供应紧张而大量增加进口。在出口方面,全球肉类和肉制品需求的大部分增长由巴西、欧洲联盟、美国、阿根廷、泰国和加拿大来满足。此外,全球鱼类和鱼类产品贸易2019年预计将下降1.2\%。

\begin{table}[]
\centering
\begin{threeparttable}
\textbf{\caption{2019年世界畜禽产品生产、消费、库存和贸易}}
\small
\setlength{\tabcolsep}{8.5mm}
\begin{tabular}{@{}lcccc@{}}
\hline
       &2017年 & 2018年 & 2019年 & 2019年同比增长率 \\ 
\hline
生产量 &333.6 &338.6& 335.2& -1.0\% \\
~~~~牛肉  &69.6  &71.3  &72.2  &1.3\% \\
~~~~禽类  &122.3 &124.6 &130.5 &14.7\% \\
~~~~猪肉  &119.8 &120.7 &110.5 &-8.5\% \\
~~~~羊肉  &15.2  &15.3  &15.4  &0.8\% \\
贸易量 &32.7 & 33.8  &36.0 & 6.7\% \\
~~~~牛肉  &9.8 &10.5  &11.1  &6.0\% \\
~~~~禽类  &13.2 & 13.5 & 14.1 & 4.4\% \\
~~~~猪肉  &8.2 &8.4 &9.4 &12.2\% \\
~~~~羊肉  &1.0 &1.0 &1.0 &-1.2\% \\
FAO 价格指数
(2002-2004=100)& 170 &166 &173 &3.6\% \\
\hline
\end{tabular}
\begin{tablenotes}
\item \tiny{资料来源:联合国粮食及农业组织(FAO)2019年11月发布的《粮食展望》。}
\end{tablenotes}
\end{threeparttable}
\end{table}

\subsection{2020年世界农业产业走势展望}
\textbf{2020年,新冠肺炎疫情导致全球经济增速放缓,OECD下调全球经济增长预期至2009年以来的最低水平,农产品需求长期趋弱。}加上近期东非和西南亚地区沙漠蝗灾情较重,可能引发全球资本市场和农产品市场恐慌,加剧国际农产品价格震荡,给农产品贸易带来极大的不确定性。在生产方面,根据经济合作与发展组织(OECD)的数据预测,2020年全球小麦产量将达2.9亿吨,比2019年增长0.4\%;全球玉米产量将达5亿吨,比2019年增长1.2\%。全球大米库存量继续降低;全球肉类、奶制品等产量持续增加;全球谷物和肉类产量保持上升趋势,高于同期人口增长速度,人均占有量稳步提高。

\section{2019年中国农业产业走势与2020年展望}
\subsection{2019年中国农业产业走势回顾}
\textbf{一、农业发展稳中有进、稳中向好}

\textbf{农产品生产保持稳定。}2019年粮食生产创历史新高,产量达到66384万吨,连续5年站稳1.3万亿斤台阶。棉油糖生产保持稳定,果菜茶供应充足。2019年以来,受猪周期、非洲猪瘟疫情、部分地区不合理禁限养等因素叠加影响,生猪产能下降较多。但随着国家和地方一系列生猪稳产保供政策措施密集出台、落地,养殖场户增养补栏信心增强,全国生猪生产止降回升。前三季度猪肉产量3181万吨,同比下降17.2\%;三季度末生猪存栏30675万头,同比下降28.5\%。禽肉、牛羊肉增产明显,其中,牛羊肉增加35万吨,禽肉增加300万吨,禽蛋产量较上一年增长5.8\%,鸡蛋产量超2700万吨,为弥补猪肉供应短缺起到关键作用。渔业生产产业结构持续优化,前三季度,全国水产品总量4165.50万吨,同比增长0.64\%。

\textbf{市场运行基本平稳。}2019年,稻谷、小麦市场价格偏弱运行。玉米产需有缺口、库存仍充裕,价格小幅波动。受生猪产能下降影响,猪用玉米饲料消费减少,但肉禽等替代品饲料消费增长明显,玉米饲料消费总体小幅下降。大豆市场购销两旺,价格稳中走强。猪肉市场供给持续偏紧,上半年价格涨幅较大,下半年市场预期趋稳,猪肉价格有所回落后保持总体稳定,受猪价上涨带动,牛羊禽肉、禽蛋、牛奶价格总体走强,但这些产品市场供给充足,涨幅明显低于猪肉。

\textbf{农产品贸易继续增长。}2019年,我国农产品进出口额2300.7亿美元,同比增5.7\%,其中,出口791.0亿美元,减1.7\%;进口1509.7亿美元,增10.0\%;贸易逆差718.7亿美元,增26.5\%。全年谷物进口1791.8万吨,同比减12.6\%;棉花进口193.7万吨,同比增19.0\%;食糖进口339.0万吨,同比增21.3\%;食用植物油进口1152.7万吨,同比增42.5\%;畜产品进口同比增27.0\%,其中,猪肉进口199.4万吨,增67.2\%;水产品进口同比增25.6\%。蔬菜和水果出现贸易逆差,进口同比增长15.9\%、23.2\%。

\textbf{二、农业供给侧结构性改革初见成效,农业高质量发展进一步提升}
   
   \textbf{继续优化调整农业结构。}2019年,全国粮食品种结构持续优化。大豆振兴计划实现良好开局,大豆种植面积增加1380多万亩,粮食播种面积减少,油菜籽、花生、蔬菜等经济作物播种面积较往年有所增加,粮改饲面积达到1500万亩。农业区域布局持续优化,江淮赤霉病高发区、华北地下水超采区和西南条锈病菌源区通过休耕和轮作等措施调减冬小麦播种面积;非优势区的稻谷、玉米播种面积进一步调减,生产进一步向优势区域集中。2019年,稻谷种植面积2969万公顷,减少50万公顷;小麦种植面积2373万公顷,减少53万公顷;玉米种植面积4128万公顷,减少85万公顷;稻谷产量20918万吨,减产0.2\%;小麦产量13374万吨,增产2\%;玉米产量26077万吨,增产1\%。
    
    \textbf{持续推进农业绿色发展。}化肥农药施用量连续第3年负增长,三大主粮化肥利用率达到39.2\%,农药利用率达到39.8\%。整建制推进585个畜牧大县畜禽粪污资源化利用,实施东北地区秸秆处理和西北地区农膜回收行动,秸秆、畜禽粪污综合利用率分别达到85\%、74\%,西北地区农膜回收率近80\%。耕地轮作休耕试点面积扩大到3000万亩,启动实施长江重点流域禁捕。果树茶有机肥替代化肥试点县增加到175个,已覆盖29个省份。新认定41个国家农业绿色发展先行区。新颁布实施农兽药残留限量标准1152项,全年农产品质量安全例行监测合格率达到97.4\%。
    
    \textbf{大力推进科技强农。}2019年,农业科技进步贡献率达到59.2\%,全国农作物耕种收综合机械化率超过70\%,主要农作物自主选育品种提高到95\%以上,新创建主要农作物全程机械化示范县151个。

\textbf{三、农村一二三产业融合水平不断提高,新型农业经营主体稳步发展}

    \textbf{新型经营主体稳步发展。}2019年,全国农业产业化龙头企业8.7万家,国家重点龙头企业1542家,其中2019年新认定农业产业化国家重点龙头企业299家。国家重点龙头企业实力不断增强,企业平均总资产规模、销售收入均超过8亿元,平均辐射带动农户2万户左右。全国家庭农场超过70万家,农民合作社达到220万家,全年托管服务面积12亿亩次,服务小农户6000万户。同时,加强培育农业产业化联合体,加快探索形成龙头企业牵引、农民合作社和家庭农产跟进、广大农户参与的产业化联合体。加强数字农业农村建设,建设运营益农信息社30多万个。

    \textbf{农村一二三产业融合速度加快。}2019年,加工产值持续增加,建设15.6万座初步加工设施,新增初加工能力1000万吨,果蔬农产品产后损失率从15\%降至6\%。加工产能向主产区和优势区布局,重心向大众城市郊区、加工园区、产业集聚区和物流节点下沉。副产物综合利用水平不断提升,农产品物流骨干网络和冷链物流体系建设不断完善。农村一二三产融合发展水平提升,推动农业不断纵向延伸、横向发展,呈现多种“农业+”态势,建设各类乡村产业园1万多个,稻渔综合种养面积超过3000万亩,农村网络销售额突破1.5万亿元,乡村旅游接到游客32亿人次。特色供给不断增加。因地制宜发展特色种养、特色食品、特色编织等乡土产业,新增全国“一村一品”示范村镇422个,总计2851个。大力实施品牌提升行动,建立中国农业品牌目录制度,累计认定绿色、有机和地理标志农产品4.3万个。

    \textbf{产业扶贫带动效果明显。}2019年,贫困地区发展初加工,累计减损增收约20亿元。贫困地区培育农业产业化龙头企业1.4万家、农民合作社61万个。开展产销对接,带动贫困地区销售农产品超过500亿元。


\subsection{2020年中国农业产业走势展望}
\textbf{一、粮食作物:产量持平略增,进口压力加大}

\textbf{中国粮食生产基本面良好,预计粮食产量将达到6.7亿吨。}根据中国农业产业模型(China Agricultural Sector Model)预测,2020年稻谷、小麦、玉米产量将分别达到2.1亿吨、1.3亿吨和2.6亿吨。粮食种植面积较2019年略有减少,达到11628万公顷,其中小麦种植面积2364万公顷,稻谷种植面积2956万公顷,玉米种植面积4147万公顷。随着大豆振兴计划的开展,大豆产量预计达到1847.47万吨,国产大豆供给能力稳步提升。在贸易方面,中美贸易磋商达成协议后,粮食进口压力将加大。预计2020年中国粮食净进口量将突破1亿吨,其中大豆净进口量将达到8853万吨。

\textbf{二、经济作物:供需基本保持稳定,品种结构有所调整}

\textbf{2020年,随着农业供给侧结构性改革的深入,经济作物品种结构有所调整。}生产方面,预计2020年蔬菜种植面积为2039万公顷,较2019年略有下降。棉花目标价格政策继续执行,种植面积和产量较2019年均有下降。油菜籽、食糖、花生、水果其他主要经济作物,油菜籽、食糖、花生、水果等面积和产量与2019年相比基本持平。消费方面,2020年经济作物的加工需求继续增加,人均食用消费量基本保持稳定。贸易方面,受国内需求增长影响,2020年食糖、水果进口量继续快速增加,蔬菜出口量下降66\%。

\textbf{三、畜牧水产:供需相对处于紧平衡,猪肉生产仍存在较大不确定性}

\textbf{2019年大部分畜牧水产品供需相对处于紧平衡,猪肉生产仍存在较大不确定性。}生产方面,预计2020年牛肉、鸡肉、奶类、水产品产量与2019年相比增幅均超过2\%,其中鸡肉产量增幅最大,预计将达到7\%,禽蛋和羊肉产量较上年分别下降6\%和1\%。受非洲猪瘟疫情影响,猪肉生产仍存在较大的不确定性,预测产量较上年降低10\%。消费方面,大部分畜产品和水产品食用消费量均呈增长趋势。预计2020年,中国牛肉、鸡肉和奶类的人均食用消费量同比分别增加4\%、6\%和2\%。羊肉和禽蛋消费量基本保持稳定。受市场价格的影响,猪肉消费持续低迷,预计较上一年下降9\%。贸易方面,与2019年相比,2020年猪肉(24\%)、牛肉(5\%)和水产品(9\%)的净进口量均有较大幅度的增加,增幅分别达到24\%、5\%和9\%。鸡肉净进口量较上一年下降70\%。



\chapter{重大事件回顾}
\begin{titledbox}{\pfhei{主要观点}}
\begin{itemize}
  \item 2019年中美贸易摩擦持续、非洲猪瘟疫情蔓延、草地贪夜蛾肆虐给中国农业产业带来严重冲击。
  \item 5G赋能智慧农业、区块链技术获认可、农业品牌目录制度建设为中国农业发展带来新机遇。
 \end{itemize}
\end{titledbox}

回望2019年,世界形势复杂严峻,我国外部环境和内在条件发生深刻变化,众多重大事件使农业发展增加了不确定性,总体呈“稳中有变、变中有忧、忧中有喜”的特点。

\section{中美贸易摩擦持续}
在2019年9月举行的中美副部长级谈判上,主要集中在农业方面,包括美国要求中国大幅增加购买美国大豆和其他农产品。在中美贸易结构中,农产品贸易既是中美经贸合作最早开始的领域,也是美国对中国具有出口优势的重要双边贸易领域。中美之间不仅农业贸易互补性较强而且农产品贸易在双方市场上均占据重要地位。尽管中国是农业大国,但离农业强国还有上升空间,有些农产品进口依赖程度高且渠道单一。因此,中国在该次贸易战中主要以对美进口农产品征税作为博弈筹码或反击举措,然而这难免会对中国农产品进出口贸易总量、贸易结构和国内农业生产、农产品价格及其替代品价格等造成不同程度的影响。

\section{非洲猪瘟疫情蔓延}
非洲猪瘟病毒抵抗力强、传播途径多,容易扩散蔓延。自2018年8月国内通报首例非洲猪瘟疫情,2019年“疫情”在全国诸多地方快速爆发,后期虽得到有效控制,但猪肉疯狂涨价,为此国务院出台了抑制猪价的政策,即“取消超出法律法规的生猪禁养、限养规定,发展规模养殖,支持农户养猪”。总体而言,非洲猪瘟对生猪产业的影响呈“危”“机”并存态势,具体可细分为以下六个方面:一是无法适应现代工业化养殖模式的传统养殖户被率先淘汰;二是无法适应现代高强度疫病防控体系的养殖场均面临生存危机;三是“南猪北养”区域转移放缓;四是国内养殖密度大且缺乏相关疫苗及兽药,致使猪价涨幅高、持续时间长(即猪周期长);五是为降低疫情管理成本和感染风险,生猪养殖模式或从“自繁自养”(一体化养殖)向“公司+农户”(分散化养殖)转变;六是防疫技术与养殖管理壁垒大幅提高,加速行业集中度提升。

\section{草地贪夜蛾肆虐}
草地贪夜蛾也称秋黏虫,原分布于美洲热带和亚热带地区,是联合国粮农组织全球预警的迁飞性农业重大害虫,具有适生区域广、寄主范围宽、迁飞能力强、繁殖倍数高、扩散速度快、暴食危害重、防控难度大等特点,主要为害玉米、水稻、甘蔗、高粱等禾本科植物,也危害十字花科、葫芦科、茄科等80余种植物,对国家农业及粮食生产安全具有严重威胁。2019年1月,草地贪夜蛾传入我国云南省,及至4月下旬,虫情扩散明显加快。截至5月,全国有13省(区)61个市(州)261县(市、区)查见幼虫为害玉米,发生面积为108万亩。据农业农村部统计,全年有25个省份发现草地贪夜蛾,见虫面积1500多万亩,实际危害面积246万亩。据有关研究测算,草地贪夜蛾在中国的适生区占全国面积约52.79\%,其中高、中、低度适生区分别占全国面积4.75\%、12.14\%、35.90\% 。中高度适生区包括华南、华中、华东、西南地区东部、陕西局部、云南局部和台湾局部;低度适生区包括黄淮海和北方玉米区。

\section{5G赋能智慧农业}
2019年6月,工信部正式向运营商发放5G商用牌照,标志着中国正式进入5G商用元年。5G全称第五代移动通信系统,是一种大幅度的技术升级,具备高速率、低时延、广连接三大新特性,可聚焦农业生产、经营、管理、服务四个环节,为农民提供实时数据以监控、跟踪及自动化其农业系统,从而达到提高盈利能力、效率和安全性的目的。目前,由于农村地区在很大程度上仍缺乏可用的网络,致使可用的技术不够先进,还无法应对智能农业所需的大量数据和速度。但是,随着诸项条件的逐步成熟,5G技术将推动移动互联网、物联网、大数据、云计算、人工智能等关联领域裂变式发展,并在种植技术智能化、农业管理智能化、种植过程公开化、劳力管理智能化以及助推特色农产品等方面为农业带来翻天覆地的变化,由此大幅提升农业全链条发展效率,助力我国农业向网络化、智慧化、数字化、精准化转型。

\section{区块链技术获认可}
2019年10月24日,中共中央政治局就区块链技术发展现状和趋势进行第十八次集体学习,这是迄今区块链技术获得的最高级别认可。区块链是由多个独立节点参与的分布式数据库系统,具有不易修改、不易伪造、可追溯的特点,区块链能够对所发生过的各种信息进行记录。在国家倾力支持的形势下,区块链农业将迎来爆发期。具体而言,区块链技术在农业中的主要应用场景包括农产品溯源、农业物联网、农业安全生产、农业大数据、农产品供应链以及农村金融与保险等。其功能集中体现在以下方面:一是帮助消费者快速了解农产品生产流通过程,防止农业生产者生产假冒伪劣产品。二是保证农业流通公司在区块链平台上进行的各种交易真实可信,有利于建立良好信用,降低交易成本。三是监控和记录农业生产、流通环节的所有信息,减少索赔,降低农业保险企业成本。

\section{农业品牌目录制度建设启动}
为提升我国农业品牌竞争力,培育一批“中国第一,世界有名”的农业品牌,我国农业品牌目录制度建设启动发布会于2019年5月10日在全国农业展览馆召开。品牌化是衡量农业现代化的核心标志,对推进社会消费升级、农业转型升级、促进农民增收和提高农产品市场竞争力等均具有正向影响。加快推进农业品牌建设,已成为转变农业发展方式,加快推进现代农业的一项紧迫任务。农业品牌目录制度建设是系统推进农产品品牌建设和品牌农业发展的一个抓手,是依据相关评价标准分类遴选国家级农业品牌形成目录,并对目录品牌进行推介、管理和保护的制度安排。随着我国农业品牌化战略的深入实施,建立农产品品牌目录制度有助于推动市场主体做大做强农产品品牌,不断优化品种、提升品质,提升农业供给体系的质量和效率,增强消费者对农产品品牌的信心,推动我国农业由“数量优势”向“品牌优势”转变。

\part{热点主题}

\chapter{效率视角的中国农业产业竞争力}
\begin{titledbox}{\pfhei{主要观点}}
\begin{itemize}
  \item 主要计算、分析历年中国和主要产区主要农产品全要素生产率。
 \end{itemize}
\end{titledbox}
预计3月31日完成。

\chapter{贸易视角的中国农业产业竞争力}
\begin{titledbox}{\pfhei{主要观点}}
\begin{itemize}
  \item 主要计算、分析历年中国和主要国家主要农产品比较优势。
 \end{itemize}
\end{titledbox}


\chapter{成本视角的中国农业产业竞争力}
\begin{titledbox}{\pfhei{主要观点}}
\begin{itemize}
  \item 主要计算、分析历年中国、主要产区和主要国家主要农产品成本。
 \end{itemize}
\end{titledbox}
预计4月3日完成

\chapter{生猪生产恢复对中国农业产业的影响}
\begin{titledbox}{\pfhei{主要观点}}
\begin{itemize}
  \item 模拟分析非洲猪瘟疫情得到有效控制、生猪生产恢复对中国农业产业的影响。
 \end{itemize}
\end{titledbox}
预计3月31日完成

\chapter{草地贪夜蛾对中国玉米产业的影响}
\begin{titledbox}{\pfhei{主要观点}}
\begin{itemize}
  \item 根据现有数据和研究结论,全国低方案、中方案、高方案和最高方案下玉米单产的损失率分别为:0.40\%、1.17\%、2.60\%和6.94\%。
  \item 开放市场下,假设没有草地贪夜蛾影响,2020年中国玉米产量预计为26422.67万吨,净进口量预计为305.19万吨;草地贪夜蛾导致2020年中国玉米产量降至24588.94~26316.98万吨的区间,玉米净进口量增至410.88~2138.92万吨的区间。
  \item 封闭市场下,草地贪夜蛾导致2020年中国玉米价格较没有草地贪夜蛾影响下的基准方案提高0.82\%~17.85\%,播种面积较基准方案提高0.31\%~6.44\%,产量较基准方案下降0.09\%~0.92\%,饲料需求较基准方案下降0.09\%~0.26\%,加工需求较基准方案下降0.08\%~1.63\%,食用需求较基准方案下降0.18\%~3.31\%。
  \item 建议建立并完善草地贪夜蛾防控工作机制,将危害损失降到最低;建立草地贪夜蛾虫害预警监测信息发布体系,实现信息共享;稳定主产区玉米播种面积,保障国内玉米安全供给。
 \end{itemize}
\end{titledbox}

\section{草地贪夜蛾对中国玉米产业的影响现状}
\subsection{草地贪夜蛾的入侵与影响过程}
草地贪夜蛾是起源于美洲热带和亚热带地区的多食性害虫,2016年从非洲开始,迅速在撒哈拉以南的44个国家蔓延,所到之处玉米、甘蔗等作物严重减产。2018年草地贪夜蛾陆续在亚洲向北、向南扩展传入印度、孟加拉国、尼泊尔、斯里兰卡、缅甸、老挝、泰国、越南等南亚和东南亚16个国家,后从云南入侵中国。

2019年1月11日草地贪夜蛾在中国云南首发之后,呈现由南到北、由西到东的扩散特点。1月份草地贪夜蛾在云南省内迅速蔓延,云南省普洱市等3个市(州)11个县(市)冬玉米田受害,发生面共计6715亩;4月份相继入侵中国广西、贵州、广东、湖南5个省的112个县区,玉米发生面积超过12.74万亩,虫情进入快速扩大为害期;5月份已扩散到海南、福建、浙江、湖北、四川、江西、重庆、河南等13个省的261个县区,玉米发生面积108万亩,整个长江流域都出现了草地贪夜蛾,农业农村部在此时召开了全国草地贪夜蛾防控工作视频会议,全面安排部署草地贪夜蛾防控工作;6月继续向北扩散,但由于中国防控措施得当,北迁趋势得到抑制;9月,农业农村部宣布草地贪夜蛾对玉米主产区的威胁全面解除;截至10月8日,草地贪夜蛾已侵入中国西南、华南、江南、长江中下游、黄淮、西北、华北地区的26省1518个县,玉米发生面积为1598.13万亩,其他作物发生面积为22.63万亩。2020年3月6日,草地贪夜蛾在云南、广东、海南、广西、福建、四川、贵州、江西8省(区)228个县见成虫,玉米田累计发生面积76万亩,云南近期边境站点出现虫量突增现象。目前发生县数接近去年5月初水平,总体发生时间比去年早2个月左右。

\begin{table}[]
\centering
\begin{threeparttable}
\textbf{\caption{草地贪夜蛾入侵与影响过程}}
\setlength{\tabcolsep}{4mm}
\small
\begin{tabular}{lll}
\hline
时间  & 入侵范围 & 造成影响   \\
\hline
2019年1月 &云南省普洱市等3个市(州)11个县(市)  &发生面积共计6715亩  \\ 
2019年4月 &云南、广西、贵州、广东、湖南5省(区)  & 玉米发生面积超过12.74万亩 \\
 & 29个市(州)112个县(市、区)& \\
2019年5月 &云南、广西、贵州、广东、湖南、海南、福建、浙江、湖北、  &玉米发生面积108万亩 \\ 
& 四川、江西、重庆、河南等13个省(区)261个县(市、区)& \\
2019年10月 & 西南、华南、江南、长江中下游、黄淮、西北、华北地区 &玉米发生面积1598.13万亩, \\ 
 & 的26省(区)1518个县(市、区) &其他作物发生面积22.63万亩 \\
2020年3月 & 云南、广东、海南、广西、福建、四川、贵州、江西8省(区)  &玉米累计发生面积76万亩 \\ 
 & 228个县(市、区)& \\
\hline
\end{tabular}
\begin{tablenotes}
\item \tiny{资料来源:根据公开资料整理。}
\end{tablenotes}
\end{threeparttable}
\end{table}

\subsection{草地贪夜蛾对玉米的危害}

\textbf{草地贪夜蛾在玉米整个生长时期均可为害,玉米从苗期至籽粒期均可受害。}首见为害期以苗期至喇叭口期最为普遍,心叶、茎秆、雄穗、花丝、雌穗等都是其钻蛀为害的重点部位。在苗期,草地贪夜蛾幼虫会钻蛀幼苗基部,造成幼苗死亡,缺苗断垄,它还可以吃光整株玉米的叶片,仅剩光杆;在穗期,直接危害玉米穗;在花粒期,能够直接咬食玉米籽粒,给玉米的产量和品质造成很大影响。

\textbf{草地贪夜蛾主要以幼虫取食危害,不同幼龄虫子所食玉米部位有差异。}1~3龄幼虫属于低龄幼虫,通常潜伏在新叶背面取食,主要食用叶片单侧的表皮和叶肉部位,留下半透明“窗孔”的上表皮,同时幼虫会吐丝,借助风力危害周围植株。4龄后进入暴食期,取食叶片形成不规则的长形孔洞。4~6龄幼虫属于高龄幼虫,常隐藏在玉米心叶中取食心叶,严重时能吃完整株玉米叶片,可造成玉米生长点死亡。且高龄幼虫还钻蛀玉米茎杆,取食玉米雄穗和果穗,并会扩散至周围植株继续危害玉米生长,导致玉米减产甚至绝收。

\subsection{草地贪夜蛾对玉米生产的影响}
\textbf{一、影响区域:黄淮海及北方玉米主产区需重点防范}

草地贪夜蛾在中国的发生区域可分为南部周年繁殖区、长江流域迁飞过渡区、黄淮海及北方重点防范区。南部周年繁殖区主要是云南、广西、广东和海南,其玉米面积并不大,播种面积和产量占中国玉米的5\%左右,因此这个区域草地贪夜蛾主要是繁殖累计数量,为全国提供虫源。长江流域迁飞过渡区主要包括湖南、湖北、重庆、四川、浙江等地区,其玉米播种面积和产量占中国玉米的10\%左右,草地贪夜蛾在此会繁殖一代或做短暂停留继续北迁,很有可能会对中国玉米产量构成很大的威胁。黄淮海及北方玉米主产区主要包括河南、河北、北京、天津、山东、山西、江苏、黑龙江、吉林、辽宁等地区,是北方春玉米区和黄河夏玉米区,还有少量的西北灌溉玉米区,其玉米总的播种面积和产量占中国玉米的60\%~70\%,一旦发生草地贪夜蛾危害,尽管是少量的危害,也可能对中国玉米的安全生产构成比较大的危害。

\textbf{二、发生面积:2019年蔓延至26个省1524个县,2020年发生面积可能更大}

2019年全国26个省(市、区)1524个县(区、市)见虫,22个省份查见幼虫,查实发生面积1688万亩,其中玉米发生面积占98.1\%。2019年草地贪夜蛾对云南、广西、湖北、四川和湖南的玉米面积影响较大,尤其是云南所受影响远远大于其他省份。总的来看,2019年受草地贪夜蛾影响最大的是西南山地玉米区,其中云南有930万亩,占全国见虫面积的60\%;广西约200万亩,占12\%;四川110多万亩,占7\%,这三省的见虫面积占全国八成左右,西南丘陵玉米区和南方山地玉米区玉米损失分别控制在5\%和3\%之内,黄淮海夏玉米区有点片零星发生,基本没有造成损失。

2020年境内外虫源的双重叠加,在一定程度上加重了中国草地贪夜蛾的发生程度,且2020年春季北迁时间提早1个月,蔓延到黄淮海、东北地区的迁飞时间也将提前,草地贪夜蛾有可能在4月初到达长江流域,5月到黄河流域,6月迁飞至东北地区。农业农村部在2020年2月发布的《2020年全国草地贪夜蛾防控预案》中,预计2020年发生面积在1亿亩左右,而且虫情呈“越冬量更大、北迁时间更早、发生区域更广、危害程度更重”特点,2020年中国草地贪夜蛾重发态势明显、形势严峻。

%\begin{figure}
%\centering
%\includegraphics[scale=0.6]{cornaa.pdf}
%\caption{ 2019年1—9月草地贪夜蛾对22个省份玉米面积的影响情况}
%\begin{tablenotes}
%\item \tiny{资料来源:姜玉英等(2019)。}
%\end{tablenotes}
%\end{figure}

\textbf{三、产量影响:2019年影响较小,但2020年的影响不可忽视}

目前侵入中国的草地贪夜蛾主要是“玉米型”,对玉米苗期到成熟期均有危害,因此草地贪夜蛾对玉米产量影响较大。在美国佛罗里达草原,草地贪夜蛾的平均损失可以减产20\%,严重的时候可以高达32\%,在经济落后的地区,草地贪夜蛾的危害造成的产量损失更高。草地贪夜蛾在拉美、洪都拉斯造成的减产达到40\%,在阿富汗达到45\%~60\%,南美没有转基因玉米种植之前的产量损失也是非常大的,其危害可造成阿根廷和巴西分别减产72\%和34\%。国际农业科研中心在2017年报道,草地贪夜蛾在当时入侵的12个非洲国家中造成的产量损失是830万~2000多万吨,仅在东非的埃埃塞俄比亚、坦桑尼亚、乌干达三国的实际玉米产量损失就高达620万吨,这个产量损失的数字相当于中国玉米每年病毒害产量造成的损失(郭井菲等,2018)。

2019年草地贪夜蛾在中国是试探性进攻,种群数量需要一个积累的过程,且中国采取了非常有力的措施,在长江流域十分有效地控制了草地贪夜蛾的种群数量,南方玉米产区产量损失控制在5\%以内,黄淮海等玉米主产区没有造成损失,实现了“防虫害稳秋粮”的目标。2020年草地贪夜蛾会早迁一代,在长江流域繁殖一代后有可能会在5月下旬到6上旬迁飞到黄淮夏玉米地区,甚至到东北春玉米区,此时正是黄淮夏玉米的苗期和东北春玉米的早期,是草地贪夜蛾的危害关键时期,一旦迁入北方重点防御区,若虫情得不到及时有效的控制,其危害将明显重于2019年,因此2020年草地贪夜蛾对玉米产量的影响不。

\textbf{四、经济影响:如不采取不防治情况下将造成较大经济损失}

草地贪夜蛾具有寄主植物种类多、迁飞能力强、繁殖潜力大、抗药性强的特点,所到之处不仅危害作物,还会给当地经济带来一定的损失,2017年造成入侵的12个非洲国家经济损失达24亿~61亿美元。国内学者利用不同的模型预测出了草地贪夜蛾可能对中国造成的经济损失。秦誉嘉等(2020)利用随机模型@RISK预测出,防治场景下草地贪夜蛾对中国玉米的潜在经济损失总量可以降低到不防治场景下的12\%左右,中国平均投入66.67亿元的防治成本可挽回1183.03亿元的经济损失,约为防治成本的17.74倍。王磊等(2019)通过建立入侵省级区域数量、县级区域数量、入侵危害面积与入侵时间长度关系模型得出,假设草地贪夜蛾在不防治或者防治效果不良情况下中国经济损失可能达100亿元。

\textbf{五、其他影响:危害作物种类多,化学防控对环境影响较大}

草地贪夜蛾危害作物种类较多对,对水稻构成危害的风险较小,但可以取食危害玉米、小麦、大麦、高粱、甘蔗、大豆、花生、油菜、向日葵、香蕉、蔬菜等多种农作物。在种群密度较低的情况下,主要危害玉米,但当种群密度较大或者没有玉米的情况下,将对小麦等作物生产造成影响。

草地贪夜蛾具有较强的抗药性,会在一定程度上增加农药的使用剂量和次数,过多的喷施农药可能会杀死天敌昆虫,从而引起环境污染和生态失衡等问题,同时也会对作物的品质和安全构成威胁。

\subsection{草地贪夜蛾的防控手段与效果}
\textbf{一、防控目标}

农业农村部在2020年2月发布的《2020年全国草地贪夜蛾防控预案》(下简称《预案》)中提出了总体防控目标和区域防控目标。总体目标:实现“两个确保”,即确保虫口密度达标区域应防尽防,确保发生区域不大面积成灾。防控处置率90\%以上,总体危害损失控制在5\%以内。区域目标:西南华南周年繁殖区,虫口密度达标区域防控处置率95\%以上,危害损失率控制在8\%以内。江南江淮迁飞过渡区,虫口密度达标区域防控处置率90\%以上,危害损失率控制在5\%以内。黄淮海及北方重点防范区,虫口密度达标区域防控处置率85\%以上,危害损失率控制在3\%以内。

\textbf{二、防控策略与效果}

(一)现阶段:采取以化学防治为主的综合应急防控策略综合应急防控策略,实施分区治理与联防联控相结合

综合应急防控策略可使用性诱杀、灯光诱杀和田间调查等办法对草地贪夜蛾进行种群监测,同时采用化学防治、生物防治、嗜好作物诱杀等手段进行田间应急防控工作。根据草地贪夜蛾在中国发生的不同区域可实行分区治理与联防联控相结合的措施(杨普云等,2019)。南部周年繁殖区要借助植被种类多、天地昆虫资源丰富的优势,充分利用生态防控,降低春季向北迁飞的虫源基数;长江流域迁飞过渡区要重点提高春末夏初防控效率和效果,压低迁入虫源基数,控制虫源迁出数量;夏季以后,黄淮海及北方玉米重点防范区要加强虫情监测,利用幼虫防治和成虫诱杀技术,做好应急防控措施,防止出现因大面积成灾威胁粮食安全问题。同时要实施联防联控,要加强与境外虫源地区的信息交流,提高防控能力与效果。

取得的效果:2019年中国采取了入侵前预警、监测,入侵后防控等措施,玉米受灾面积比预期减少了1亿多亩,黄淮海等玉米主产区没有造成损失,打赢了“虫口夺粮”攻坚战,保障了玉米安全生产,赢得了全年粮食和农业丰收主动权。

(二)长远:构建绿色可持续控制技术体系

化学防控在短期内可取得较好的防控效果,但长期来看不利于低成本、绿色可持续的发展目标。因此草地贪夜蛾的防控工作需要发展种群迁飞监测预测、成虫迁移阻截和幼虫高效控制关键技术(吴孔明,2020)。目前中国昆虫雷达的应用技术相对较成熟,已初步具备构建国家昆虫迁飞雷达监测网络的基础,通过雷达网络能够精确定位并定量草地贪,夜蛾的成虫迁移动态,利用灯诱、食诱等手段降低成虫的发生密度。Bt 玉米对草地贪夜蛾幼虫的控制效率高于90\%,发展Bt玉米能够从源头阻截草地贪夜蛾幼虫,保护玉米生产,当下中国Bt玉米对草地贪夜蛾的抗性评价工作已经取得良好效果,要尽快制定以Bt玉米为核心的草地贪夜蛾绿色可持续控制技术体系。

\section{草地贪夜蛾对2020年中国玉米供需影响的模拟方案设置}
模拟方案的设置基于农业农村部《预案》对草地贪夜蛾影响的基准判断,在总结现有草地贪夜蛾影响范围和影响程度的基础上,将影响面积和影响程度折算为全国年度玉米单产降幅,利用CASM模拟草地贪夜蛾对我国玉米和其他主要作物供需的影响。

\subsection{各省影响面积模拟方案}
2019 年草地贪夜蛾完成了在中国的入侵和定殖过程,2020 年始将进入暴发为害阶段,将在春、夏两季随季风向北逐代迁移进入中国长江流域、黄河流域和东北地区(吴孔明,2020)。根据Wang等(2019)基于MaxEnt模型的分析,全国有116.16万公顷的面积在气候上最适合草地贪夜蛾生存,并列出了各省的分析结果。Li等(2019)基于Weather Research and Forecasting模型预测,2020年6月,草地贪夜蛾将进入黄淮海玉米主产区,7月将进入东北玉米主产区。南京农业大学植物保护学院胡高教授在接受《中国科学报》访问时提出,草地贪夜蛾很有可能在2020年4~5月迁入长江流域,5~6月到达黄河流域,7月到达东北地区。与《预案》相比,专家的研究结论表明东北地区也有可能受到草地贪夜蛾影响。

基于Wang等(2019),本研究根据下式计算得到了草地贪夜蛾可能对各省份造成影响的国土面积比重($w_i$)。

\[w_i=(0*\Sigma_iA_{1i}+0.25*\Sigma_iA_{2i}+0.5*\Sigma_iA_{3i}+0.75*\Sigma_iA_{4i}+\Sigma_iA_{5i})/\Sigma_{j=1}^5A_{ji}\]

其中,$A_1$、$A_2$、$A_3$、$A_4$和$A_5$分别为Wang等(2019)测算的不适合草地贪夜蛾生存的面积、较不适合的面积、适合的面积、较适合的面积和最适合面积,$i$为各省。

假设适合草地贪夜蛾生存的国土与适合玉米种植的国土重合,基于$w_i$和国家统计局公布的2018年全国各省玉米播种面积($AA_i$),可以得到理论上各省玉米生产区出现草地贪夜蛾的权重($wcorn_i$)。

\[wcorn_i=w_i*AA_i/(\Sigma_i (w_i*AA_i))\]

根据《预案》,设定草地贪夜蛾发生面积1亿亩为中方案($SA_2$),0.5亿亩为低方案($SA_1$),1.5亿亩为高方案($SA_3$),2亿亩为最高方案($SA_4$),那么不同模拟方案下,各省草地贪夜蛾的玉米发生面积($FAWA_{ik}$)为:

\[FAWA_{ik}=wcorn_i*SA_k\]

其中,$k=1,2,3,4$。

\begin{table}[]
\centering
\begin{threeparttable}
\textbf{\caption{各省影响面积模拟方案}}
\setlength{\tabcolsep}{4.5mm}
\small
\begin{tabular}{lrrrrrrr}
\hline
\multicolumn{1}{c}{地区} & \multicolumn{1}{c}{$w_i$} & \multicolumn{1}{c}{$AA_i*$} & \multicolumn{1}{c}{$wcorn_i$} & \multicolumn{1}{c}{$FAWA_{i1}*$} & \multicolumn{1}{c}{$FAWA_{i2}*$} & \multicolumn{1}{c}{$FAWA_{i3}*$} & \multicolumn{1}{c}{$FAWA_{i4}*$} \\
\hline
北京	&20.78\%	&40.09		&0.05\%	&2.49 	&4.97 	&7.46 	&9.95 \\
天津	&24.80\%	&186.77		&0.28\%	&13.82 	&27.65 	&41.47 	&55.29 \\
河北	&18.37\%	&3437.74	&3.77\%	&188.50 &377.00 &565.51 &754.01 \\
山西	&24.73\%	&1747.67	&2.58\%	&129.02 &258.05 &387.07 &516.09 \\
内蒙古&	0.86\%	&3742.14	&0.19\%	&9.62 	&19.25 	&28.87 	&38.50 \\
辽宁	&11.47\%	&2712.98	&1.86\%	&92.85 	&185.69 &278.54 &371.38 \\
吉林	&0.12\%		&4231.47	&0.03\%	&1.48 	&2.97 	&4.45 	&5.93 \\
黑龙江&	0.00\%	&6317.82	&0.00\%	&0.00 	&0.00 	&0.00 	&0.00 \\
上海	&99.07\%	&1.81		&0.01\%	&0.54 	&1.07 	&1.61 	&2.14 \\
江苏	&62.77\%	&515.77		&1.93\%	&96.64 	&193.28 &289.92	&386.56 \\
浙江	&97.48\%	&49.34		&0.29\%	&14.36 	&28.71 	&43.07 	&57.42 \\
安徽	&86.21\%	&1138.56	&5.86\%	&292.97 &585.94	&878.91 &1171.88 \\
福建	&99.63\%	&28.79		&0.17\%	&8.56 	&17.12 	&25.68 	&34.25 \\
江西	&99.07\%	&35.00		&0.21\%	&10.35 	&20.70 	&31.05 	&41.40 \\
山东	&36.86\%	&3934.68	&8.66\%	&432.93 &865.85	&1298.78&1731.70 \\
河南	&62.48\%	&3918.96	&14.62\%&730.90 &1461.81&2192.71&2923.61 \\
湖北	&77.08\%	&781.20		&3.59\%	&179.73 &359.45 &539.18	&	718.91 \\
湖南	&91.70\%	&359.20		&1.97\%	&98.31 	&196.62 &294.94 &393.25 \\
广东	&98.56\%	&120.08		&0.71\%	&35.33 	&70.65 	&105.98 &141.30 \\
广西	&94.53\%	&584.43		&3.30\%	&164.90 &329.81	&494.71 &659.61 \\
海南	&96.40\%	&			&		&		&		&		&	\\
重庆	&73.55\%	&442.33		&1.94\%	&97.10 	&194.20 &291.30 &388.41 \\
四川	&28.85\%	&1856.00	&3.20\%	&159.85 &319.70 &479.55 &639.40 \\
贵州	&56.31\%	&602.12		&2.02\%	&101.21 &202.41 &303.62 &404.82 \\
云南	&56.02\%	&1785.20	&5.97\%	&298.52 &597.05 &895.57 &1194.09 \\
西藏	&1.27\%		&5.16		&0.00\%	&0.02 	&0.04 	&0.06 	&0.08 \\
陕西	&37.70\%	&1179.47	&2.65\%	&132.73 &265.46 &398.19 &530.92 \\
甘肃	&8.04\%		&1012.74	&0.49\%	&24.29 	&48.59 	&72.88 	&97.17 \\
青海	&0.23\%		&18.45		&0.00\%	&0.01 	&0.03 	&0.04 	&0.05 \\
宁夏	&16.78\%	&310.79		&0.31\%	&15.56 	&31.13 	&46.69 	&62.25 \\
新疆	&0.24\%		&1033.29	&0.01\%	&0.74 	&1.48 	&2.21 	&2.95 \\
\hline
\end{tabular}
\begin{tablenotes}
\item \tiny{资料来源:姜玉英等(2019)、Wang等(2019)、国家统计局。}
\item \tiny{注:*表示单位为千公顷。}
\end{tablenotes}
\end{threeparttable}
\end{table}

\subsection{各省影响面积模拟方案}

草地贪夜蛾对各省玉米生产的影响程度用单产的降幅来衡量。假设姜玉英等(2019)给出的2019年各省份草地贪夜蛾玉米被害株率($PIP_i$)等同于玉米单产降幅。根据姜玉英等(2019)给出的统计数据,各省份草地贪夜蛾玉米被害株率与草地贪夜蛾在该省首见月份存在相关性。因此,根据2019年各省份草地贪夜蛾玉米被害株率和Li等(2019)预测的2020年各省首见月份推算得到4种模拟方案下的2020年各省份玉米单产降幅($YDL_i$)。

低方案($SY_1$)假设$YDL_i=minPIP_{im}*YD_i$,中方案($SY_2$)假设$YDL_i=meanPIP_{im}*YD_i$,高方案($SY_3$)假设$YDL_i=maxPIP_{im}*YD_i$,最高方案($SY_4$)假设$YDL_i=2*maxPIP_{im}*YD_i$。其中,m为首见月份,$YD_i$为2018年各省玉米单产,$min$为最小值,$mean$为均值,$max$为最大值。

%\begin{figure}
%\centering
%\includegraphics[scale=1]{cornbb.pdf}
%\caption{2019年玉米被害株率与首见月份}
%\begin{tablenotes}
%\item \tiny{资料来源:姜玉英等(2019)。}
%\end{tablenotes}
%\end{figure}

\begin{table}[]
\centering
\begin{threeparttable}
\textbf{\caption{各省影响程度模拟方案}}
\setlength{\tabcolsep}{3.75mm}
\small
\begin{tabular}{lrrrrrrrr}
\hline
\multicolumn{1}{c}{地区} & \multicolumn{1}{c}{$PIP_i$} & \multicolumn{1}{c}{2019首} & \multicolumn{1}{c}{2020首} & \multicolumn{1}{c}{$YD_{i}*$} & \multicolumn{1}{c}{$YDL_{i1}*$}& \multicolumn{1}{c}{$YDL_{i2}*$} & \multicolumn{1}{c}{$YDL_{i3}*$} & \multicolumn{1}{c}{$YDL_{i4}*$} \\
\multicolumn{1}{c}{} & \multicolumn{1}{c}{} & \multicolumn{1}{c}{见月份} & \multicolumn{1}{c}{见月份} & \multicolumn{1}{c}{} & \multicolumn{1}{c}{}& \multicolumn{1}{c}{} & \multicolumn{1}{c}{} & \multicolumn{1}{c}{} \\
\hline
北京  &         & 8 & 5~6 & 6770.24 & 88.01  & 345.90  & 765.04  & 1530.07 \\
天津  &         & 9 & 5~6 & 5919.22 & 76.95  & 302.42  & 668.87  & 1337.74 \\
河北  & 0.10\%  & 8 & 5~6 & 5646.59 & 73.41  & 288.49  & 638.06  & 1276.13 \\
山西  & 3.00\%  & 7 & 5~6 & 5616.75 & 73.02  & 286.96  & 634.69  & 1269.39 \\
内蒙古 &         & 8 & 5~6 & 7215.00 & 93.80  & 368.62  & 815.30  & 1630.59 \\
辽宁  &         &   & 5~6 & 6129.00 & 79.68  & 313.14  & 692.58  & 1385.15 \\
吉林  &         &   & 5~6 & 6616.80 & 86.02  & 338.06  & 747.70  & 1495.40 \\
黑龙江 &         &   & 7   & 6303.05 &        &         &         &         \\
上海  & 2.60\%  & 5 & 3~4 & 6930.18 & 492.04 & 776.18  & 1254.36 & 2508.73 \\
江苏  & 3.70\%  & 5 & 3~4 & 5815.58 & 412.91 & 651.34  & 1052.62 & 2105.24 \\
浙江  & 5.10\%  & 5 & 3~4 & 4183.20 & 297.01 & 468.52  & 757.16  & 1514.32 \\
安徽  & 2.40\%  & 5 & 3~4 & 5231.26 & 371.42 & 585.90  & 946.86  & 1893.72 \\
福建  & 7.10\%  & 5 & 1   & 4361.50 & 645.50 & 806.88  & 968.25  & 1936.51 \\
江西  & 6.00\%  & 5 & 1   & 4471.43 & 317.47 & 500.80  & 809.33  & 1618.66 \\
山东  & 1.50\%  & 7 & 4~5 & 6626.10 & 470.45 & 742.12  & 1199.32 & 2398.65 \\
河南  & 1.30\%  & 5 & 4~5 & 6000.01 & 426.00 & 672.00  & 1086.00 & 2172.00 \\
湖北  & 5.10\%  & 5 & 3~4 & 4139.53 & 612.65 & 765.81  & 918.98  & 1837.95 \\
湖南  & 7.10\%  & 4 & 3~4 & 5646.44 & 835.67 & 1044.59 & 1253.51 & 2507.02 \\
广东  & 10.60\% & 4 & 1   & 4541.97 & 672.21 & 840.26  & 1008.32 & 2016.63 \\
广西  & 11.40\% & 4 & 1   & 4678.06 & 692.35 & 865.44  & 1038.53 & 2077.06 \\
海南  & 18.10\% & 4 & 1   &         &        &         &         &         \\
重庆  & 4.20\%  & 5 & 3~4 & 5681.88 & 403.41 & 636.37  & 1028.42 & 2056.84 \\
四川  & 6.10\%  & 5 & 1   & 5745.15 & 850.28 & 1062.85 & 1275.42 & 2550.85 \\
贵州  & 8.80\%  & 4 & 1   & 4300.80 & 636.52 & 795.65  & 954.78  & 1909.56 \\
云南  & 18.50\% & 1 & 1   & 5187.09 & 767.69 & 959.61  & 1151.53 & 2303.07 \\
西藏  & 11.30\% & 5 & 3~4 & 6517.84 & 84.73  & 333.00  & 736.52  & 1473.03 \\
陕西  & 3.90\%  & 5 & 4~5 & 4952.64 & 351.64 & 554.70  & 896.43  & 1792.86 \\
甘肃  & 6.40\%  & 7 & 5~6 & 5825.70 & 75.73  & 297.64  & 658.30  & 1316.61 \\
青海  &         &   &     & 6249.32 &        &         &         &         \\
宁夏  &         & 7 & 5~6 & 7549.15 & 98.14  & 385.69  & 853.05  & 1706.11 \\
新疆  &         &   &     & 8009.10 &        &         &         &    \\
\hline
\end{tabular}
\begin{tablenotes}
\item \tiny{资料来源:姜玉英等(2019)、Li等(2019)、国家统计局。}
\item \tiny{注:*表示单位为千克/公顷;2020年首见月份为1表示越冬区域。}
\end{tablenotes}
\end{threeparttable}
\end{table}

\subsection{用于CASM模型的全国模拟方案}
基于影响面积和影响程度,可以得到用于中国农业产业模型(CASM)的全国模拟方案。基准方案($S_0$)下,2020年的全国玉米产量$PD=AA*YD$,其中$AA$为全国玉米播种面积,$YD$为全国玉米单产。假设全国玉米播种面积不受草地贪夜蛾的影响,那么全国玉米产量的损失率等于玉米单产的损失率($YSL$)。低方案($S_1$)、中方案($S_2$)、高方案($S_3$)和最高方案($S_4$)下玉米单产的损失率($YSL$)分别为:

\[YSL_k=\Sigma_i (FAWA_{ik}*YDL_k)/PD\]

其中,$k=1,2,3,4$。

基准方案($S_0$)下,$YSL_0=0.00\%$。经过测算,$YSL_1=0.40\%$,$YSL_2=1.17\%$,$YSL_3=2.60\%$,$YSL_4=6.94\%$。

\section{模拟结果}
\subsection{开放市场下的模拟结果}
假设国内玉米市场开放,即可以进行自由的国际贸易,那么国内玉米价格等同于国际市场价格,并且不会发生变化,因而国内需求也不发生变化。中国玉米单产和产量下降导致玉米净进口量增加,且净进口量的增幅等于产量的降幅。具体模拟结果为:

第一,在基准方案($S_0$)下,假设没有草地贪夜蛾影响,2020年中国玉米产量预计为26422.67万吨。草地贪夜蛾导致2020年中国玉米产量降至24588.94~26316.98万吨的区间。

第二,在基准方案($S_0$)下,假设没有草地贪夜蛾影响,2020年中国玉米净进口量预计为305.19万吨。草地贪夜蛾导致2020年中国玉米净进口量增至410.88~2138.92万吨的区间。

\begin{table}[]
\centering
\begin{threeparttable}
\textbf{\caption{开放市场下草地贪夜蛾对中国玉米供需的影响}}
\setlength{\tabcolsep}{2.6mm}
\small
\begin{tabular}{lrrrrr}
\hline
\multicolumn{1}{c}{} & \multicolumn{1}{c}{基准方案($S_0$)} & \multicolumn{1}{c}{$低方案(S_1)$} & \multicolumn{1}{c}{$中方案(S_2)$
} & \multicolumn{1}{c}{$高方案(S_3)$} & \multicolumn{1}{c}{$最高方案(S_4)$}  \\
\hline
冲击因素:单产指数     & 100.00   & 99.60    & 98.83    & 97.40    & 93.06    \\
\hline
价格指数          & 100.00   & 100.00   & 100.00   & 100.00   & 100.00   \\
播种面积指数        & 100.00   & 100.00   & 100.00   & 100.00   & 100.00   \\
产量指数          & 100.00   & 99.60    & 98.83    & 97.40    & 93.06    \\
国内需求指数        & 100.00   & 100.00   & 100.00   & 100.00   & 100.00   \\
\hline
2020年产量(万吨)   & 26422.67 & 26316.98 & 26113.52 & 25735.58 & 24588.94 \\
2020年净进口量(万吨) & 305.19   & 410.88   & 614.34   & 992.18   & 2138.92    \\
\hline
\end{tabular}
\begin{tablenotes}
\item \tiny{资料来源:中国农业产业模型(CASM)。}
\end{tablenotes}
\end{threeparttable}
\end{table}

%\begin{figure}
%\centering
%\includegraphics[scale=1]{corncc.pdf}
%\caption{开放市场下草地贪夜蛾对中国玉米产量的影响}
%\begin{tablenotes}
%\item \tiny{资料来源:国家统计局、中国农业产业模型(CASM)。}
%\item \tiny{注:*代表预测值。}
%\end{tablenotes}
%\end{figure}

\subsection{封闭市场下的模拟结果}

假设国内玉米市场封闭,那么国内玉米价格由国内玉米供求决定,不受国际市场影响。中国玉米单产和产量下降导致玉米价格上升,进而导致播种面积增加,需求减少。具体模拟结果为:

第一,草地贪夜蛾引起的玉米减产导致国内玉米价格提高。草地贪夜蛾导致2020年中国玉米价格较没有草地贪夜蛾影响下的基准方案提高0.82\%~17.85\%。

第二,玉米价格增长导致播种面积增加。草地贪夜蛾导致2020年中国玉米播种面积较没有草地贪夜蛾影响下的基准方案提高0.31\%~6.44\%。

第三,单产下降和播种面积增加的综合效果是玉米产量微幅下降。草地贪夜蛾导致2020年中国玉米产量较没有草地贪夜蛾影响下的基准方案下降0.09\%~0.92\%。

第四,玉米价格增长导致需求减少。草地贪夜蛾导致2020年中国玉米饲料需求较没有草地贪夜蛾影响下的基准方案下降0.09\%~0.26\%,加工需求较没有草地贪夜蛾影响下的基准方案下降0.08\%~1.63\%,食用需求较没有草地贪夜蛾影响下的基准方案下降0.18\%~3.31\%。

\begin{table}[]
\centering
\begin{threeparttable}
\textbf{\caption{开放市场下草地贪夜蛾对中国玉米供需的影响}}
\setlength{\tabcolsep}{2.6mm}
\small
\begin{tabular}{lrrrrr}
\hline
\multicolumn{1}{c}{} & \multicolumn{1}{c}{基准方案($S_0$)} & \multicolumn{1}{c}{$低方案(S_1)$} & \multicolumn{1}{c}{$中方案(S_2)$
} & \multicolumn{1}{c}{$高方案(S_3)$} & \multicolumn{1}{c}{$最高方案(S_4)$}  \\
\hline
冲击因素:单产指数     & 100.00   & 99.60    & 98.83    & 97.40    & 93.06    \\
\hline
价格指数   & 100.00 & 100.82 & 102.64 & 106.14 & 117.85 \\
播种面积指数 & 100.00 & 100.31 & 100.99 & 102.29 & 106.44 \\
产量指数   & 100.00 & 99.91  & 99.82  & 99.64  & 99.08  \\
饲料需求指数 & 100.00 & 99.92  & 99.90  & 99.86  & 99.74  \\
加工需求指数 & 100.00 & 99.92  & 99.74  & 99.41  & 98.37  \\
食用需求指数 & 100.00 & 99.82  & 99.46  & 98.78  & 96.69    \\
\hline
\end{tabular}
\begin{tablenotes}
\item \tiny{资料来源:中国农业产业模型(CASM)。}
\end{tablenotes}
\end{threeparttable}
\end{table}

%\begin{figure}
%\centering
%\includegraphics[scale=1]{corndd.pdf}
%\caption{封闭市场下草地贪夜蛾对中国玉米价格指数的影响}
%\begin{tablenotes}
%\item \tiny{资料来源:国家统计局、中国农业产业模型(CASM)。}
%\item \tiny{注:基准方案价格指数为100。}
%\end{tablenotes}
%\end{figure}

\section{对策建议}

\textbf{第一,建立并完善草地贪夜蛾防控工作机制,将危害损失降到最低}
针对中国草地贪夜蛾的为害现状,本着“长短结合、标本兼治”的原则,建立并完善“应急防控与长期防控相结合、区域防控与联防联控相结合”的工作机制,推进实施专业化的统防统治行动,将危害损失降到最低,保证国家粮食安全。一要建立以政府主导的防控机制,将草地贪夜蛾防控工作上升到国家战略的地位。成立以政府主管领导挂帅的草地贪夜蛾防控指挥部,上升到政府行为,从思想上统一认识、高度重视。二要建立完善属地管理、联防联控的工作协调机制。针对草地贪夜蛾在中国主产区局部地区突发性特点,当前要以应急防控为主,建立和完善应急防控预案、物资储备、队伍组建及快速反应相互配套的应急防控机制;针对虫害跨区域迁飞性特点,要建立区域间的防控协调机构,开展信息共享、技术协作、配合密切、责任明确的区域联防联控行动。三是要实施推进专业化的统防统治行动。针对中国目前“小农生产”的特点,一家一户防病治虫难、成本高,要积极探索和推进专业化统防统治的有效模式;国家要增加对统防统治的经费投入,从财政投入上予以最大的支持。四要加快草地贪夜蛾防控关键技术的突破。当前以化学防控为主的应急防控已经取得良好的防治效果,但是长期来看还要因地制宜,建立以生态调控为基础、生物防治为核心、应急化学农药防控为辅的综合防治技术体系(郭井菲,2019)。防治技术要产品研发上突破,要集中力量进行推广。

\textbf{第二,建立草地贪夜蛾虫害预警监测信息发布体系,实现信息共享}
全面完善草地贪夜蛾预警监测系统中数据采集上传、数据挖掘利用、预报展示发布等主要工作环节,借助物联网、人工智能、云计算等现代高新技术,建立虫害预警监测信息发布体系,使草地贪夜蛾虫害预警监测体系进入发展进步的快车道,达到数据采集自动化、数据上传网络化、数据分析智能化、预测模型化、展示图形化、发布多元化的目标。建立草地贪夜蛾预警监测信息发布平台或渠道,确保信息公开顺畅发布,实现信息共享,加快制定信息共享机制,充分利用广播、电视、智能手机、公众号、互联网等新媒体,向全社会公开发布草地贪夜蛾监测信息,实现信息共享,把信息及时传递到农民手中,破解最后一公里的困境,以便更有效的打赢这场“虫口夺粮”战斗。

\textbf{第三,稳定主产区玉米播种面积,保障国内玉米安全供给}
从当前中国玉米供需形势的总体判断来看,供需形势发生逆转,由于玉米面积调减产量的下降,未来生猪产能的回复以及工业需求产能的扩张,产区缺口将由扩大的趋势。2020年如果草地贪夜蛾呈重发态势,将会对我国粮食安全带来重大风险。因此当前要清醒地意识到保障国内玉米安全供给的重要性。一要谨慎调减玉米面积,稳定国内玉米供给。要稳定玉米优势产区的播种面积,提高玉米种植者生产积极性,进一步完善生产者补贴制度,补贴要向优势产区倾斜。二要依靠科技提高国内玉米单产,应大力促进科技创新,培育优质高产的玉米良种,增强抵御自然灾害的能力。

\part{产业前瞻}
\chapter{谷物产业}
\begin{titledbox}{\pfhei{主要观点}}
\begin{itemize}
  \item 2019年,种植结构调整继续推进,稻谷播种面积和产量持续下降;小麦、玉米种植面积下降,单产及总产增长。稻谷消费平稳略增,小麦消费小幅增长,玉米需求整体放缓。稻米出口9年来首次超过进口,小麦、玉米进口呈增长态势。
  \item 2020—2021年,稻谷播种面积和总产量稳中略减,小麦产量基本保持稳定;玉米播种面积恢复性增长,单产和总产稳步增加。稻谷、玉米消费小幅增长,小麦消费先降后增。稻谷出口增速放缓,小麦净进口减少,玉米净进口大幅增加。
  \item 小麦、玉米国际市场竞争不强,稻米国际竞争优势近年有所提升。
 \end{itemize}
\end{titledbox}

本章主要针对稻谷、小麦、玉米2019年产业发展新动态、2020—2021年供求形势及近年国际竞争力进行判断。

\section{2019年产业发展新动态}
\subsection{生产形势}
\textbf{种植结构调整继续推进,稻谷播种面积和产量持续下降。}随着农业供给侧结构性改革深入推进,稻谷种植结构进一步优化,播种面积和产量连续两年出现下降。国家统计局数据显示,2019年中国稻谷播种面积为2969.40万公顷,同比减少49.55万公顷,减幅1.64\%;单位面积产量7059.00公斤/公顷,同比增长32.41公斤/公顷,增幅0.46\%;总产量20961.00万吨,同比减少251.90万吨,下降1.19\%。从种植结构看,低质低效的早稻和非优势区的稻谷播种面积持续调减,优质高效单季稻播种面积持续增加。其中,早稻播种面积为445万公顷,同比减少34.1万公顷,减幅为7.1\%;早稻总产量2627万吨,同比减产232.5万吨,下降8.1\%。

\textbf{小麦种植面积小幅下降,总产和单产不同程度增长。}国家统计局数据显示,2019年中国小麦种植面积为2372.70万公顷,比2018年减少53.90万公顷,下降2.22\%;小麦总产量为13359.00万吨,比2018年增加215.00万吨,增长1.64\%;小麦单产为5630.29千克/公顷,比2018年增加213.66千克/公顷,增长3.94\%。随着农业供给侧结构性改革,小麦在种植结构、区域布局及品种结构方面进行了优化调整。且自去年秋播以来,由于气象条件总体有利,小麦在种植面积小幅下降的情况下,总产和单产均实现恢复性增长。此外,小麦总体质量也较2018年有显著提升。根据国家粮食和物资储备局发布的《2019年新收获小麦质量调查情况的报告》,2019年主产省小麦整体质量较好。对河北、山西、江苏、安徽、山东、河南、湖北、陕西等8个主产省的样品检测结果显示,与2018年相比,一等比例增加31.5\%,三等以上比例增加10.4\%。

\textbf{玉米播种面积连续四年减少,玉米单产及总产量呈增长态势。}随着农业供给侧结构性改革的推进及大豆振兴计划的实施,玉米播种面积受到一定程度压减。国家统计局数据显示,与2016年相比,2019年玉米播种面积减少289.36万公顷,产量减少284.31万吨。2019年玉米播种面积为4128.40万公顷,比上年减少84.60万公顷,减幅2.01\%,主要是东北地区种植结构调整,减少玉米种植,改种大豆。玉米单产6316.49千克/公顷,比上年增加212.19千克/公顷,增幅3.48\%;玉米产量26077.00万吨,较上年增加359.60万吨,增幅1.40\%,产量增加主要由单产水平提高推动。

\begin{table}[]
\centering
\begin{threeparttable}
\textbf{\caption{2018—2019年中国三大谷物生产情况变化}}
\setlength{\tabcolsep}{2.8mm}
\small
\begin{tabular}{lrrrrrrrrr}
\hline
\multirow{2}{*}{品种} & \multicolumn{3}{c}{播种面积(万公顷)} & \multicolumn{3}{c}{单产(千克/公顷)} & \multicolumn{3}{c}{总产量(万吨)}    \\ 
 & 2018年    & 2019年    & 同比变化    & 2018年     & 2019年    & 同比变化   & 2018年    & 2019年    & 同比变化    \\ 
\hline
稻谷   &3,018.95    &2,969.40   &-1.64\%   &7,026.59    &7,059.00  & 0.46\%  &21,212.90  & 20,961.00 & -1.19\% \\
小麦   &2,426.60    &2,372.70   &-2.22\%   &5,416.63    &5,630.29  & 3.94\%  &13,144.00  & 13,359.00 & 1.64\% \\
玉米   &4,213.00    &4,128.40   &-2.01\%   &6,104.30    &6,316.49  & 3.48\%  &25,717.40  & 26,077.00 & 1.40\% \\
合计   &9,658.55    &9,470.50   &-1.95\%   &6,219.81    &6,377.38  & 2.53\%  &60,074.30  & 60,397.00 & 0.54\% \\
\hline
\end{tabular}
\begin{tablenotes}
\item \tiny{资料来源:国家统计局。}
\end{tablenotes}
\end{threeparttable}
\end{table}

\subsection{消费形势}
\textbf{稻谷消费平稳略增,工业用粮增幅较大。}稻谷消费弹性小、产业链较短等特点,决定了消费总体较为稳定,需求增量有限。随着人民生活水平持续提高,人均主食消费量保持下降趋势,稻谷年度食用消费稳中略降;在去库存政策背景下,部分不宜食用稻谷进入工业和饲料消费领域,年度工业消费量增加较明显;受非洲猪瘟疫情影响,年度饲料消费量有所下降。根据中国农业产业模型(CASM),2019年稻谷食用消费为15826.16万吨,同比略减23.84万吨,减幅0.15\%;工业消费为2043.62万吨,同比增加293.62万吨,增幅16.78\%;饲料消费为539.75万吨,同比减少111.73万吨,减幅17.15\%。

\textbf{小麦消费总量小幅上升,工业消费增幅明显。}随着城镇化率的提高,国内居民消费结构发生明显变化。国内居民人均粮食消费量逐渐减少。根据中国农业产业模型(CASM),2019年中国小麦食用消费量继续减少。2019年小麦食用消费为9256.10万吨,比上年减少23.90万吨,减幅 0.26\%。一方面,由于新小麦质量普遍好于去年,可以达到国标二级以上水平,小麦的饲料消费出现下滑。另一方面,由于非洲猪瘟等原因导致生猪存栏大幅下降,猪肉消费明显下降,小麦的饲料消费也随之减少。2019年小麦饲料消费为1473.03万吨,比上年减少298.41万吨,减幅 16.85\%。工业消费方面,在产能扩张支撑以及去库存的背景下,2019年小麦工业消费需求上涨,工业消费量达到1529.00万吨,比上年增加329.00万吨,增长27.42\%。

\textbf{受非洲猪瘟疫情影响,玉米需求整体放缓。}国家临储玉米成交量、成交率均比去年同期下降,用粮企业和贸易商采购临储玉米积极性明显下降。饲用消费方面,因非洲猪瘟等原因导致生猪存栏大幅下降已经成为事实,猪肉消费明显下降,玉米饲用消费随之减少。根据中国农业产业模型(CASM),2019年饲料消费14915.76万吨,较上年下降9.44\%。工业消费方面,在产能扩张支撑下,玉米工业消费需求增加。2019年玉米工业消费8491.28万吨,较上年增加8.86\%。燃料乙醇将成为下一个增长极,产量缺口空间较大。产能继续大幅增长将抵消部分行业开工率下降和稻谷替代玉米生产燃料乙醇的影响,玉米工业消费将继续增长。

\begin{table}[]
\centering
\begin{threeparttable}
\textbf{\caption{2018—2019年中国三大谷物需求情况变化}}
\setlength{\tabcolsep}{7mm}
\small
\begin{tabular}{llrrrr}
\hline
\multicolumn{1}{c}{} & \multicolumn{1}{c}{} & \multicolumn{1}{c}{稻谷} & \multicolumn{1}{c}{小麦} & \multicolumn{1}{c}{玉米} & \multicolumn{1}{c}{合计} \\
\hline
总需求(万吨)  & 2018年 & 21,354.43 & 13,425.37 & 26,163.90 & 60,943.70 \\
         & 2019年 & 20,961.00 & 13,676.50 & 26,553.40 & 61,190.90 \\
         & 同比变化  & -1.84\%   & 1.87\%    & 1.49\%    & 0.41\%    \\
食用需求(万吨) & 2018年 & 15,850.00 & 9,280.00  & 1,865.00  & 26,995.00 \\
         & 2019年 & 15,826.16 & 9,256.10  & 1,858.96  & 26,941.22 \\
         & 同比变化  & -0.15\%   & -0.26\%   & -0.32\%   & -0.20\%   \\
饲料需求(万吨) & 2018年 & 651.48    & 1,771.44  & 16,471.30 & 18,894.22 \\
         & 2019年 & 539.75    & 1,473.03  & 14,915.76 & 16,928.54 \\
         & 同比变化  & -17.15\%  & -16.85\%  & -9.44\%   & -10.40\%  \\
工业需求(万吨) & 2018年 & 1,750.00  & 1,200.00  & 7,800.00  & 10,750.00 \\
         & 2019年 & 2,043.62  & 1,529.00  & 8,491.28  & 12,063.89 \\
         & 同比变化  & 16.78\%   & 27.42\%   & 8.86\%    & 12.22\% \\
\hline
\end{tabular}
\begin{tablenotes}
\item \tiny{资料来源:中国农业产业模型(CASM)。}
\end{tablenotes}
\end{threeparttable}
\end{table}

\subsection{贸易形势}
\textbf{稻米出口9年来首次超过进口。}2019 年,中国稻谷和大米进出口贸易结构发生逆转,自2011年以来出口量首次超过进口量。海关数据显示,2019年中国累计进口稻米255万吨,同比减幅17.2\%,从2017年进口的历史高点连续第二年下降;累计出口稻米275万吨,同比增幅31.5\%,达到2004年以来出口的最高水平。从进出口流向看,中国大米最主要的进口国是越南、泰国,两者占比近八成;出口国主要是埃及、科特迪瓦、土耳其、朝鲜、巴布亚新几内亚等。中国稻米进口量下降、出口量增幅明显,主要是受近年来国家去库存力度加大及下调稻谷最低收购价、降低政策性稻谷底价政策的影响,国内大米价格总体水平下降,与东南亚大米价差大幅缩小,进口大米价格优势减弱,国产大米国际竞争力不断增强。

\textbf{
小麦进口增长,进口来源呈多元化。}海关数据显示,2019年中国进口小麦348.79万吨,比上年增长12.54\%;出口小麦31.32万吨,比上年增长9.66\%。2019年中国小麦进口多元化趋势明显,2019年中国从美国、加拿大、澳大利亚等国进口小麦占比有所下降,从哈萨克斯坦、立陶宛进口小麦占比有所提高。受中美贸易摩擦影响,2019年中国从美国进口小麦占比下滑至3.8\%,比2018年减少71\%。

\textbf{玉米进口呈增长态势。}受产需缺口扩大影响,进口玉米将成为常态;同时中美贸易摩擦对国内玉米进口与供需关系带来一定影响。一是进口国家结构发生变化,但玉米进口总量并未减少。海关数据显示,中美贸易摩擦后中国自美国进口的玉米大幅减少,降幅为61\%,自乌克兰进口的玉米则增长了62\%。二是玉米替代品进口由于失去价格优势而大幅减少,对国内玉米需求是一个促进因素。2019年,中国进口玉米479.1万吨,上年进口量为352.2万吨,同比增加126.9万吨,增幅36\%;出口玉米2.57万吨,同比增加1.37万吨,增幅为114\%。进口来源国主要为乌克兰、美国、老挝,其中乌克兰进口占比92.4\%、美国占比6.8\%。

\subsection{产业政策}
\textbf{一、稻谷产业}

\textbf{一是继续调整完善稻谷收储制度和补贴政策。}2019年中央1号文件明确提出将稻谷作为必保品种。国家继续深化稻谷收储制度和价格形成机制改革,继续在稻谷主产区实行最低收购价政策,继续在有关稻谷主产省实施水稻种植者补贴、水稻轮作休耕补贴等政策,确保农民种粮收益基本稳定。经过2016—2018年连续调低稻谷最低收购价后,稻谷最低收购价基本回落至2011—2012年水平,2019年稻谷最低收购价保持稳定,维持早籼稻(三等,下同)、中晚籼稻和粳稻最低收购价格每50公斤120元、126元和130元不变。同时,国家继续对有关稻谷主产省份给予适当补贴支持,如水稻种植者补贴、轮作休耕补贴、种粮大户补贴等,2019年黑龙江省地表水灌溉稻谷亩补贴标准高于地下水灌溉稻谷40元以上(含40元),新增水稻休耕试点每亩补贴500元,地下水下降得到改善;辽宁省水稻生产者补贴金额约为80元/亩;江苏省水稻种植者补贴约为60~120元/亩。

\textbf{二是加大政策性稻谷去库存力度。}2019年稻谷持续产大于消,国内政策性稻谷库存庞大,部分省份收储压力加大。国家对稻谷市场的政策调控方向从“保障供应稳定价格”转向以“降价去库存”为重点,推动政策性稻谷不合理库存消化。2016年实行了超期储存粳稻销售新办法,2017年实行了“分年份”销售办法,2018年国家对政策性稻谷的拍卖底价进行了大幅下调,且拍卖时间较往年大幅提前。2019年国家将稻谷拍卖底价再次下调500元/吨,成交量明显增加,国家政策性稻谷竞价销售累计成交1575.1万吨。预计2020年稻谷去库存方面政策性粮食销售机制还将创新和完善,超期储存粮食预计会出台针对政策。但目前国内政策性稻谷库存达1亿吨以上,后期去库存任务仍十分艰巨。

\textbf{二、小麦产业}

\textbf{一是小麦最低收购价连续第二年下调。}最低收购价作为小麦市场价格的底部支撑,对小麦市场价格有着关键指导意义。2019年,国家继续在小麦主产区实行最低收购价政策。综合考虑粮食生产成本、市场供求、国内外市场价格和产业发展等因素,经国务院批准,2019年生产的小麦(三等)最低收购价为每50公斤112元,比2018年下调3元。在保持农民种粮收益预期基本稳定的同时,更好地发挥市场机制作用,引导粮食供给结构优化。2019年在下调小麦最低收购价同时,国家还加大“优质粮食工程”实施力度,更好地鼓励地方和农民扩大优质专用小麦等生产供给,通过优质优价实现农民增收。同时,积极探索开展粮食作物完全成本保险和收入保险试点,充分发挥农业保险在保护农民利益中的重要保障作用。

\textbf{二是政策性小麦拍卖底价下调。}2019年5月21日国家粮食交易中心发布公告称,2014—2018年最低收购价小麦拍卖底价为2290元/吨,为国标三等仓内交货价,相邻等级价差40元/吨,拍卖底价根据不同生产年份分别相比之前下降60~120元/吨。从政策指导价方面看,拍卖底价下调,有助于加快去库存进度,促使市场陈粮价格向新小麦价格靠拢,对新季小麦价格带来天花板的平抑作用。

\textbf{三、玉米产业}

\textbf{一是减少玉米生产者补贴标准,调高大豆生产者补贴。}为应对中美贸易摩擦对大豆的影响,2018年中国东北各省陆续提高大豆生产者补贴标准,黑龙江、辽宁玉米大豆补贴差额在200元以上,进一步促进农户改种大豆的意愿,政策倾斜使大豆面积继续增加,玉米面积继续调减。

\textbf{二是进一步扩大轮作补贴试点范围。}自2016年为推进农业绿色发展,中国开始实行耕地休耕轮作补贴制度,且轮作补贴和生产者补贴可以叠加,轮作补贴试点范围的扩大进一步促进了大豆种植面积增加,减少了玉米种植面积。


\begin{table}[]
\centering
\begin{threeparttable}
\textbf{\caption{2020—2021年中国三大谷物生产预测}}
\setlength{\tabcolsep}{7mm}
\small
\begin{tabular}{lrrrrrr}
\hline
\multirow{2}{*}{品种} & \multicolumn{2}{c}{播种面积(万公顷)} & \multicolumn{2}{c}{单产(千克/公顷)} & \multicolumn{2}{c}{总产量(万吨)}    \\
                    & 2020年         & 2021年         & 2020年         & 2021年         & 2020年        & 2021年        \\
\hline
稻谷   &2,955.76    &2,945.11    &7,077.08    &7,097.99    &20,918.17   &20,904.37   \\
小麦   &2,364.43    &2,352.38    &5,656.51    &5,680.46    &13,374.42   &13,362.59   \\
玉米   &4,147.38    &4,150.35    &4,147.38    &4,150.35    &26,422.67   &26,686.65   \\
合计   &9,467.57    &9,447.84    &6,412.97    &6,451.60    &60,715.25   &60,953.61   \\
\hline
\end{tabular}
\begin{tablenotes}
\item \tiny{资料来源:中国农业产业模型(CASM)。}
\end{tablenotes}
\end{threeparttable}
\end{table}

\section{2020—2021年供求形势判断}
\subsection{生产形势}

\textbf{稻谷播种面积和总产量稳中略减。}在全国粮食生产稳政策稳面积的总基调下,预计2020年稻谷生产总体保持稳定。受稻谷最低收购价以及结构调整等政策引导,2020年稻谷种植面积继续稳中适度下降,种植结构调整将继续加快,优质高效稻谷面积及产量将持续增加,低质低效早稻和非优势区稻谷面积持续调减。受科技进步、单产提高驱动,稻谷总产量持平或略有增减,产量仍可能维持在2亿吨以上,确保中国人的饭碗牢牢端在自己手里。根据中国农业产业模型(CASM),预计2020年稻谷种植面积减至2955.76万公顷,同比减少0.46\%;单产增至7077.08千克/公顷,同比增长0.26\%;总产量为20918.17万吨,同比略减0.20\%。2021年稻谷种植面积为2945.11万公顷,同比减少0.36\%;单产为7097.99千克/公顷,同比增长0.30\%;总产量为20904.37万吨,同比下降0.07\%。

\textbf{小麦产量基本保持稳定。}根据中国农业产业模型(CASM),小麦种植面积小幅减少,单产持续提高,总产量呈稳中略增态势。预计2020年,小麦种植面积略减至2364.43万公顷,同比减少0.35\%;单产提高到5656.51千克/公顷,同比增长0.47\%;总产量为13374.42万吨,同比增长0.12\%;预计2021年,小麦种植面积略减至2352.38万公顷,同比减少0.51\%;单产提高到5680.46千克/公顷,同比增长0.42\%;总产量为13362.59万吨,同比减少0.09\%。

\textbf{玉米播种面积恢复性增长,单产和总产稳步增加。}由于国内玉米连续两年出现产不足需,国家提出要稳定玉米供给,前期玉米种植面积调减步伐减慢,受价格上涨、种植利益增加影响,预计2020—2021年玉米种植面积有所增加。根据中国农业产业模型(CASM),预计2020年和2021年玉米种植面积分别达到4147.38万公顷和4150.35万公顷。由于技术进步,玉米单产稳步增加,玉米总产量将呈现增加趋势。预计2020年,玉米单产将达到6370.93千克/公顷,总产量达到26422.67万吨。

\subsection{消费形势}
\textbf{稻谷消费小幅增长。}大米食用消费总体较为稳定,随着中国全面进入小康社会、居民膳食营养结构不断优化,人均口粮消费量呈下降趋势,对优质大米的食用需求将继续增加。中国正处于政策性稻谷去库存阶段,伴随去库存力度进一步加大,大量低价稻米涌向市场,可替代部分谷物使用,预计国内稻谷工业和饲料用粮需求将增加。受非洲猪瘟疫情影响,短期内饲用消费呈现下降,随着生猪生产逐步恢复和稻米竞争力增强,中长期饲料用粮替代需求将会有所增长。根据中国农业产业模型(CASM),预计2020年稻谷国内总消费量为20923.47万吨,同比下降0.18\%,其中食用需求、饲料需求分别比上年下降0.10\%和7.81\%,工业需求增加12.65\%。2021年稻谷国内总消费量为21003.58万吨,同比增加0.38\%。其中,食用需求较上年下降0.22\%,饲料需求、工业需求较上年增加21.62\%和7.32\%。

\textbf{小麦消费需求先降后增。}受非洲猪瘟影响,预计2020年小麦饲料消费明显下降。后续随着非洲猪瘟影响的消退,生猪生产将逐步恢复,饲用需求扩张,2021年小麦饲料消费恢复性增长。根据中国农业产业模型(CASM),预计2020年小麦消费总量为13464.67万吨,比上年下降1.55\%;其中,食用消费下降0.12\%,饲料消费下降6.01\%,工业消费增长3.83\%。2021年小麦消费总量为13729.31万吨,同比增长1.97\%;其中,食用消费下降0.01\%,饲料消费增长22.43\%,工业消费增长3.79\%。

\textbf{玉米消费略有增加。}预计2020年玉米总消费较上年略有增加,其中工业消费呈现增长趋势,饲用消费减少831.10万吨。综合考虑非洲猪瘟影响及生猪产能恢复,预计2020年生猪产能恢复到80\%,受新冠肺炎疫情影响,禽类养殖业受到冲击较大,多方面因素叠加,预计2020年饲用消费减少。工业需求方面,工业需求的产能扩张态势不变,未来玉米工业需求量将继续增长。根据中国农业产业模型(CASM),2020年玉米消费总量为26727.86万吨,较上年增加0.66\%;其中,饲料消费14084.66万吨,工业消费8629.50万吨,饲料消费下降5.57\%,工业消费增长1.63\%。

\subsection{贸易形势}
\textbf{稻谷出口增速放缓。}近期政策性稻谷去库存化将继续加快,低价稻米出口竞争力较强,且可有效替代进口稻米,预计2020年中国稻米出口量仍可能增加。中长期看,随着国际大米市场竞争激烈,稻米出口增速将趋于放缓。同时,中国大米进口已连续两年大幅下降,受全球粮食市场价格和贸易政策等综合影响,预计进口降幅也将放缓。根据中国农业产业模型(CASM),预计2020年稻谷净进口量为5.30万吨,2021年净进口量达99.21万吨。

\textbf{小麦净进口减少。}受国际市场价格、贸易政策、供给侧结构性改革等影响,中国小麦净进口将进一步减少。根据中国农业产业模型(CASM),预计2020年小麦净进口量为90.26万吨,同比减少71.57\%;2021年净进口量达366.72万吨,同比增加306.31\%。

\textbf{玉米净进口呈现大幅增加。}一方面,随着玉米需求量不断增加和国内库存释放殆尽,需求的增速快于产量的增加,产需缺口进一步扩大,进口将成为常态。另一方面,随着中美贸易关系缓和,从美国进口的玉米将增加。根据中国农业产业模型(CASM),预计2020年玉米净进口量为305.19万吨,2021将突破1000万吨达到1325.25万吨。

\begin{table}[]
\centering
\begin{threeparttable}
\textbf{\caption{2020—2021年中国三大谷物需求预测}}
\setlength{\tabcolsep}{7mm}
\small
\begin{tabular}{llrrrr}
\hline
\multicolumn{1}{c}{} & \multicolumn{1}{c}{} & \multicolumn{1}{c}{稻谷} & \multicolumn{1}{c}{小麦} & \multicolumn{1}{c}{玉米} & \multicolumn{1}{c}{合计} \\
\hline
总需求(万吨)  & 2020年 & 20,923.47 & 13,464.67 & 26,727.86 & 61,116.00 \\
         & 2021年 & 21,003.58 & 13,729.31 & 28,011.90 & 62,744.79 \\
食用需求(万吨) & 2020年 & 15,809.68 & 9,244.93  & 1,851.41  & 26,906.02 \\
         & 2021年 & 15,774.85 & 9,243.96  & 1,844.33  & 26,863.14 \\
饲料需求(万吨) & 2020年 & 497.62    & 1,384.49  & 14,084.66 & 15,966.76 \\
         & 2021年 & 605.20    & 1,694.98  & 16,357.90 & 18,658.09 \\
工业需求(万吨) & 2020年 & 2,302.14  & 1,587.61  & 8,629.50  & 12,519.25 \\
         & 2021年 & 2,470.69  & 1,647.79  & 8,717.46  & 12,835.94 \\
\hline
\end{tabular}
\begin{tablenotes}
\item \tiny{资料来源:中国农业产业模型(CASM)。}
\end{tablenotes}
\end{threeparttable}
\end{table}

\section{国际竞争力分析}
\textbf{近年稻米国际竞争优势有所提升。}2000—2018年,中国稻米国际市场占有率总体波动下降,2016年以来出现较快增长,但总体水平依旧偏低,2018年为3.92\%。稻米贸易竞争力指数、显示性比较优势指数亦呈波动下降趋势,2016年后出现小幅增长。总体来说,中国稻米国际竞争力受产品品质、生产成本、国内政策及国际市场贸易环境等因素综合影响。2016年以来,随着国内稻谷持续产大于消,国家实行“去库存”政策,稳步降低最低收购价和政策性稻谷销售底价,中国稻米与东南亚稻米的价差大幅缩小,稻米进口下滑,出口增幅明显,中国稻米国际市场竞争力不断增强。就主要稻米出口国而言,近年印度稻米国际竞争力不断增强,国际市场占有率不断提升;泰国稻米品质较高,产量稳定,常年出口居世界前两位,稻米国际竞争优势较强;越南主要以出口中低端碎米为主,成本低、价格低廉是稻米具有较强国际竞争力的主要原因;巴基斯坦稻米生产成本低,明显价格优势提升了国际竞争优势;美国稻米生产总体规模小,但科技水平、规模经营、加工贸易处于世界领先水平,大米出口能力较强,但随着种植成本不断提高和东南亚国家稻米竞争日趋激烈,美国稻米国际市场占有率逐年下降,市场竞争优势减弱。

\textbf{小麦国际市场竞争优势弱。}2000—2018年,中国小麦国际市场占有率尽管部分年份有所上升,但一直在2.00\%以下,表明小麦国际市场占有率极低,小麦国际竞争力较弱。小麦贸易竞争力指数除少数年份外,均为负值,表明中国小麦国际竞争力极弱。2003—2007年中国小麦显示性比较优势指数处于0.10—0.50之间,小麦出口具备一定比较优势,但 2007年后显示性比较优势指数接近于0,表明中国小麦不再具有出口比较优势。总体来说,2000年以来中国小麦国际竞争力逐渐丧失优势。放眼国际,加拿大小麦国际竞争力较强,且常年保持稳定;俄罗斯、乌克兰小麦竞争优势迅猛增长,现处于具有强优势行列;欧盟小麦竞争力比较优势虽不明显,但国际市场占有率仍居高位;美国小麦国际竞争虽有下降,但仍具较强比较优势。

\textbf{玉米国际市场竞争力弱。}2000—2018年,中国玉米国际市场占有率尽管部分年份有所上升,但总体大幅下降,表明中国玉米贸易规模正不断缩小,玉米竞争力不断下降。2010年前中国玉米贸易竞争力指数大于0,出口还存在一定优势,但2010—2018年玉米贸易竞争力指数持续维持在-0.90左右,表明中国玉米已转变为出口竞争劣势,且净进口规模巨大。2000—2003年中国玉米显示性比较优势指数较大,2003—2007年显示性比较优势指数交替波动,2008年开始不断下降,到2018年中国玉米显示性比较优势指数仅为0.004,表明中国玉米不再具有出口比较优势。与主要玉米出口国比较,美国玉米竞争力虽存在波动,但国际市场占有率仍居高位且具有很强竞争力;阿根廷玉米竞争优势强,且常年保持稳定;巴西土地辽阔、气候适宜,玉米竞争力、国际占有率都在不断提升;乌克兰虽没有巴西、阿根廷幅员辽阔,但土壤肥沃,为玉米生产提供了巨大潜力,玉米竞争优势迅猛增长,现处于具有强优势行列;俄罗斯玉米竞争力有所增长,但与其他国家相比,仍具有明显差距。


\begin{table}[]
\centering
\begin{threeparttable}
\textbf{\caption{1995—2018年中国和主要国家稻米贸易竞争力指数和显示性比较优势指数}}
\setlength{\tabcolsep}{6mm}
\small
\begin{tabular}{lcccccc}
\hline
年份        & 中国     & 泰国     & 越南     & 印度     & 美国    & 巴基斯坦   \\
\hline
贸易竞争力指数   &        &        &        &        &       &        \\
1995      & -0.928 & 1.000  & 0.988  & 1.000  & 0.752 & 1.000  \\
2005      & 0.068  & 0.999  & 1.000  & 1.000  & 0.664 & 1.000  \\
2015      & -0.693 & 0.993  & 0.976  & 1.000  & 0.432 & 0.978  \\
2016      & -0.614 & 0.996  & 0.970  & 0.988  & 0.437 & 0.961  \\
2017      & -0.508 & 0.995  & 0.969  & 1.000  & 0.405 & 0.973  \\
2018      & -0.286 & 0.996  & /      & 1.000  & 0.273 & 0.970  \\
\hline
显示性比较优势指数 &        &        &        &        &       &        \\
1995      & 0.085  & 11.079 & 24.948 & 17.686 & 0.977 & 29.190 \\
2005      & 0.690  & 11.522 & 16.374 & 12.113 & 1.373 & 38.088 \\
2015      & 0.252  & 8.558  & 8.002  & 12.648 & 0.839 & 28.373 \\
2016      & 0.385  & 9.094  & 6.730  & 9.313  & 0.850 & 32.685 \\
2017      & 0.567  & 8.868  & 6.501  & 12.113 & 0.755 & 27.436 \\
2018      & 0.855  & 10.141 & /      & 6.679  & 0.787 & 31.391 \\
\hline
\end{tabular}
\begin{tablenotes}
\item \tiny{资料来源:基于FAOSTAT、UN Comtrade和WTO数据计算得到。}
\item \tiny{注:“/”代表缺少数据。}
\end{tablenotes}
\end{threeparttable}
\end{table}

\begin{table}[]
\centering
\begin{threeparttable}
\textbf{\caption{1995—2018年中国和主要国家(或地区)小麦贸易竞争力指数和显示性比较优势指数}}
\setlength{\tabcolsep}{6mm}
\small
\begin{tabular}{lcccccc}
\hline
年份        & 中国     & 俄罗斯    & 美国    & 加拿大   & 欧盟    & 乌克兰    \\
\hline
贸易竞争力指数   &        &        &       &       &       &        \\
1995      & -0.998 & -0.830 & 0.912 & 0.997 & 0.165 & -0.057 \\
2005      & -0.908 & 0.887  & 0.922 & 0.997 & 0.031 & 0.996  \\
2015      & -0.997 & 0.963  & 0.771 & 0.988 & 0.240 & 0.999  \\
2016      & -0.992 & 0.959  & 0.833 & 0.997 & 0.222 & 0.998  \\
2017      & -0.992 & 0.986  & 0.792 & 0.997 & 0.151 & 0.998  \\
2018      & -0.992 & 0.986  & /     & 0.995 & 0.120 & 0.998  \\
\hline
显示性比较优势指数 &        &        &       &       &       &        \\
1995      & 0.004  & 0.158  & 2.370 & 3.204 & 0.813 & 0.306  \\
2005      & 0.062  & 5.143  & 2.545 & 2.609 & 0.625 & 6.637  \\
2015      & 0.001  & 6.801  & 1.393 & 3.953 & 0.903 & 5.905  \\
2016      & 0.002  & 7.534  & 1.420 & 2.777 & 0.890 & 7.367  \\
2017      & 0.002  & 8.736  & 1.599 & 2.609 & 0.748 & 6.660  \\
2018      & 0.002  & 10.712 & /     & 3.604 & 0.699 & 6.792  \\
\hline
\end{tabular}
\begin{tablenotes}
\item \tiny{资料来源:基于FAOSTAT、UN Comtrade和WTO数据计算得到。}
\item \tiny{注:“/”代表缺少数据。}
\end{tablenotes}
\end{threeparttable}
\end{table}

\begin{table}[]
\centering
\begin{threeparttable}
\textbf{\caption{1995—2018年中国和主要国家玉米贸易竞争力指数和显示性比较优势指数}}
\setlength{\tabcolsep}{6mm}
\small
\begin{tabular}{lcccccc}
\hline
年份        & 中国     & 美国    & 巴西     & 阿根廷   & 乌克兰    & 俄罗斯    \\
\hline
贸易竞争力指数   &        &       &        &       &        &        \\
1995      & -0.968 & 0.981 & -0.943 & 0.975 & -0.928 & -0.964 \\
2005      & 0.997  & 0.945 & 0.345  & 0.992 & 0.821  & -0.695 \\
2010      & -0.834 & 0.934 & 0.930  & 0.984 & 0.690  & -0.175 \\
2016      & -0.991 & 0.897 & 0.765  & 0.974 & 0.917  & 0.723  \\
2017      & -0.941 & 0.905 & 0.913  & 0.992 & 0.915  & 0.659  \\
2018      & -0.985 & 0.942 & /      & 0.984 & 0.922  & 0.688  \\
\hline
显示性比较优势指数 &        &       &        &       &        &        \\
1995      & 0.048  & 5.069 & 0.018  & 3.256 & 0.012  & 0.023  \\
2005      & 2.886  & 4.591 & 0.261  & 5.386 & 4.291  & 0.047  \\
2010      & 0.038  & 4.200 & 1.911  & 5.376 & 2.864  & 0.162  \\
2016      & 0.002  & 3.390 & 2.635  & 5.320 & 8.994  & 1.919  \\
2017      & 0.014  & 3.284 & 3.076  & 5.386 & 9.446  & 1.752  \\
2018      & 0.004  & 4.035 & /      & 5.376 & 9.636  & 1.317    \\
\hline
\end{tabular}
\begin{tablenotes}
\item \tiny{资料来源:基于FAOSTAT、UN Comtrade和WTO数据计算得到。}
\item \tiny{注:“/”代表缺少数据。}
\end{tablenotes}
\end{threeparttable}
\end{table}

\chapter{油料作物产业}
\begin{titledbox}{\pfhei{主要观点}}
\begin{itemize}
  \item 2019年,大豆振兴计划实现良好开局,大豆生产继续回升;油菜播种面积和产量继续下降,花生产量持续增加。豆粕饲用消费大幅下降,但大豆总消费保持平稳略增;油菜籽、菜籽油、花生消费需求增加。大豆、油菜籽进口增加,花生净出口放缓。
  \item 2020—2021年,大豆播种面积及产量增加,油菜则继续保持下行趋势,花生播种面积出现波动但产量稳步提高;大豆工业需求增加推动总消费增长,油菜籽需求旺盛,花生需求小幅下降;大豆、油菜籽净进口继续扩大,花生贸易由净出口转为净进口,花生油净进口扩大,花生粕净出口下降。
  \item 大豆、油菜、花生产业国际竞争力弱,不具竞争优势。
 \end{itemize}
\end{titledbox}

本章主要针对大豆、油菜籽、花生等重要农产品的2019年产业发展新动态、2020—2021年供求形势及近年国际竞争力进行判断。

\section{2019年产业发展新动态}
\subsection{生产形势}
\textbf{大豆振兴计划实现良好开局,推动生产继续回升。}2019年,中国大豆播种面积为933.33万公顷,同比增长11.11\%;单产为1939.35千克/公顷,同比增长2.18\%;总产量为1810.00万吨,同比增长13.53\%。各地深入推进农业供给侧结构性改革,扩大大豆等优质高效作物种植规模,辽宁、吉林、黑龙江、内蒙古“三省一区”大豆面积增加量占全国增加量的九成以上。

\textbf{油菜播种面积和产量继续下降,单产水平小幅提升。}近年来,油菜籽收购价格有所上升,但上升幅度不大,而油菜种植人工成本仍然很高,致使农民种植意愿持续低迷,全国油菜播种面积和总产量仍保持下降态势。随着科技应用水平不断提高,油菜单产继续稳步提升,但提升速度有变缓倾向。根据中国农业产业模型(CASM),2019年油菜播种面积为651.31万公顷,较上年减少0.57\%;油菜产量为1321.96万吨,同比下降0.46\%;油菜单产为2029.71千克/公顷,每公顷较上年增加2.23千克,增幅为0.11\%。

\textbf{花生单产及产量持续增加。}相较于其他油料作物,花生显然是目前最适合中国大面积种植的油料作物。根据中国农业产业模型(CASM),2019年全国花生播种面积为453.49万公顷,较2018年下降2.51万公顷,同比下降0.55\%;花生总产量达1700.82万吨,较2018年增加0.82万吨,同比增长0.05\%。总体来看,为应对由中美贸易摩擦造成的大豆进口量剧减带来的油料作物缺口,花生种植技术进一步推广,单产提高是花生产量增长的主要原因。花生单产水平由2018年的3728.07千克/公顷增至2019年的3750.46千克/公顷,同比增长0.09\%。

\subsection{消费形势}

\textbf{豆粕饲用消费大幅下降,但大豆总消费保持平稳略增。}根据中国农业产业模型(CASM),2019年国内大豆总消费量为10650.00万吨,同比增加2.55\%;豆粕饲料需求量为5250.62万吨,同比下降11.47\%。受非洲猪瘟影响,生猪和能繁母猪存栏下降,猪料中豆粕需求大幅下滑。但由于居民收入水平提高和全国城镇化率增加,人们对食用植物油的需求仍未减少,使得大豆工业消费量和大豆总消费量保持平稳。

\textbf{油菜籽和菜籽油市场需求继续保持增长。}油菜籽主要用于压榨提取植物性油脂,随着城镇化水平不断提升,植物性油脂消费比重不断上升,其中菜籽油比重也逐步增加。随着菜籽油消费需求增加,国内油菜籽加工需求也随之增长。根据中国农业产业模型(CASM),2019年油菜籽加工(压榨)需求为1747.25万吨,同比增长0.56\%;菜籽油食用消费量为839.55万吨,较上年增长1.04\%。

\textbf{花生消费需求整体呈增长态势。}首先,受中美贸易摩擦影响,2019年作为中国重要油料作物的大豆进口剧减,形成对替代油料作物的极大需求。2019年,中国花生油食用需求达307.28万吨,同比增加1.51\%,由此带来的作为油料作物的花生需求出现了大幅上升,总需求量达到1708.96万吨,同比增加0.53\%。其次,由于非洲猪瘟疫情扩大及国内环保政策的双重限制,生猪等畜禽存栏大幅减少,进而推动饲料需求下滑。花生粕作为重要的饲料原料,2019年总需求量为362.23万吨,同比下降1.27\%。当前,驱动花生需求整体上升的两大因素仍在持续。一方面,尽管中美贸易谈判第一阶段协议已经签署,但不确定性无法评估,后续影响仍待观望。另一方面,居民肉类消费中猪肉以外的需求量处在上升阶段,尽管多部门联合发声保障2021年生猪产能恢复正常,但生猪产业发展仍面临诸多新问题与新挑战。

\subsection{贸易形势}
\textbf{大豆进口呈增长态势。}海关统计数据显示,2019年中国大豆净进口量为8840.00万吨,同比增加0.56\%,为历史第二高峰。近两年来,中国正拓展大豆进口的多元化渠道,扩大从南美、俄罗斯等国家的进口。受巴西大豆出口价格上升影响,2019年巴西大豆在中国全年大豆进口量中的比例下降至62\%。自2019年下半年中美贸易关系进入缓和阶段,中国进口美豆的数量开始有所增加。

\textbf{油菜籽和菜籽油净进口保持双增长。}中国是国际市场上的主要油菜籽及其制品进口国,近年油菜籽及其制品净进口保持持续增长。根据美国农业部统计,2018年中国油菜籽净进口量为348.60万吨,菜籽油净进口量为149.20万吨。根据中国农业产业模型(CASM),2019年中国油菜籽和菜籽油净进口量分别达到364.76万吨和161.00万吨,分别较上年增加4.63\%和7.91\%,继续保持双增长,且增长幅度有所提升。

\textbf{花生转为净进口国。}相较于其他油料作物,花生的自然特性更适合国内种植,促使中国长期成为花生净出口国。由于国内饮食习惯与近年餐饮服务行业的迅速扩大,对花生油的需求使得中国在花生贸易方面成为进口国。2019年,中国花生净进口量为8.14万吨。就进口国而言,从原先第一大花生进口来源国塞内加尔的进口量由2018年的31.83万吨下降至10.73万吨,进口来源更加多样。

\begin{table}[]
\centering
\begin{threeparttable}
\textbf{\caption{2018—2019年中国主要油料作物生产情况变化}}
\setlength{\tabcolsep}{2.8mm}
\small
\begin{tabular}{lrrrrrrrrr}
\hline
\multirow{2}{*}{品种} & \multicolumn{3}{c}{播种面积(万公顷)} & \multicolumn{3}{c}{单产(千克/公顷)} & \multicolumn{3}{c}{总产量(万吨)}    \\ 
 & 2018年    & 2019年    & 同比变化    & 2018年     & 2019年    & 同比变化   & 2018年    & 2019年    & 同比变化    \\ 
\hline
 大豆                  & 840.00   & 933.33   & 11.11\% & 1,897.96  & 1,939.35 & 2.18\% & 1,594.29 & 1,810.00 & 13.53\% \\
油菜籽                 & 655.06   & 651.31   & -0.57\% & 2,027.48  & 2,029.71 & 0.11\% & 1,328.12 & 1,321.96 & -0.46\% \\
花生                  & 456.00   & 453.49   & -0.55\% & 3,728.07  & 3,750.46 & 0.60\% & 1,700.00 & 1,700.82 & 0.05\%  \\
合计                  & 1,951.06 & 2,038.10 & 4.46\%  & 2,369.18  & 2,371.22 & 0.09\% & 4,622.41 & 4,832.78 & 4.55\%  \\
\hline
\end{tabular}
\begin{tablenotes}
\item \tiny{资料来源:国家统计局、中国农业产业模型(CASM)。}
\end{tablenotes}
\end{threeparttable}
\end{table}

\begin{table}[]
\centering
\begin{threeparttable}
\textbf{\caption{2018—2019年中国主要油料作物需求和贸易情况变化}}
\setlength{\tabcolsep}{2.8mm}
\small
\begin{tabular}{lrrrrrrrrrr}
\hline
\multirow{2}{*}{品种} & \multicolumn{3}{c}{食用需求(万吨)} & \multicolumn{3}{c}{工业需求(万吨)}   & \multicolumn{3}{c}{净进口量(万吨)}    \\
                    & 2018年    & 2019年    & 同比变化   & 2018年     & 2019年     & 同比变化   & 2018年    & 2019年    & 同比变化      \\
\hline
大豆                  & 708.08   & 713.35   & 0.74\% & 9,413.41  & 9,441.96  & 0.30\% & 8,791.00 & 8,840.00 & 0.56\%    \\
油菜籽                 & /        & /        & /      & 1,737.50  & 1,747.25  & 0.56\% & 348.60   & 364.76   & 4.63\%    \\
花生                  & 687.00   & 702.21   & 2.21\% & 995.00    & 1,006.75  & 1.18\% & -18.00   & 8.14     & -145.23\% \\
合计                  & 1,395.08 & 1,415.57 & 1.47\% & 12,145.91 & 12,195.96 & 0.41\% & 9,121.60 & 9,212.90 & 1.00\%  \\
\hline
\end{tabular}
\begin{tablenotes}
\item \tiny{资料来源:海关总署、中国农业产业模型(CASM)。}
\end{tablenotes}
\end{threeparttable}
\end{table}

\subsection{产业政策}
\textbf{一、大豆产业}

根据中央1号文件部署,大豆振兴计划由2019年开始稳步实施,根据农业农村部制定的实施方案,共有如下六项重点任务:

\textbf{一是完善玉米大豆生产者补贴政策。}统筹安排东北地区玉米大豆生产者补贴,按照总量稳定、结构优化原则,充分考虑平衡玉米大豆种植收益,合理确定玉米大豆具体补贴标准,充分调动农民种植大豆积极性。2019年7月,黑龙江7部门联合发布《黑龙江省2019年玉米、大豆和稻谷生产者补贴工作实施方案》,对补贴进行了详细阐述,大豆生产者补贴标准保持在每亩300元左右,高出玉米生产者补贴200元以上。

\textbf{二是完善耕地轮作试点补助政策。}东北地区主要通过玉米大豆生产者补贴标准调整引导玉米大豆轮作,将黄淮海地区和长江流域纳入轮作试点补助范围。2019年,东北地区第一批退出的500万亩轮作面积,一部分安排在黄淮海地区,支持开展夏玉米改种夏大豆或夏花生;另一部分安排在长江流域,支持开展玉米大豆轮作或间套作、水稻与油菜轮作。另外,根据农业农村部通知,2019年实施耕地轮作休耕制度试点面积3000万亩。其中,轮作试点面积2500万亩,主要在东北冷凉区、北方农牧交错区、黄淮海地区和长江流域的大豆、花生、油菜产区实施。黑龙江省对2019年新增的345.8万亩耕地轮作试点任务,全面实行米豆轮作或麦豆轮作。

\textbf{三是加快大豆高标准农田建设。}加快建设1亿亩大豆生产保护区。将高标准农田建设项目向大豆生产保护区倾斜,改善大豆生产基础条件,建成一批旱涝保收的大豆生产基地。2019年,黑龙江省农业农村厅制定了《2019年全省大豆扩种工作方案》,落实大豆高标准农田建设面积160.7万亩。推进大豆规模化种植,全省200亩以上大豆规模种植面积达到2000多万亩,在全省建立了7个省级大豆高标准科技示范园区,在大豆主产区落实大豆绿色高质高效创建示范面积555.2万亩。

\textbf{四是加大大豆良种繁育和推广力度,加快培育优良大豆新品种。}2019年,吉林省各渠道大豆品种审定试验参试品种数达到300个,其中大豆品种联合体试验、特用大豆品种自主试验参试品种达到82个。

\textbf{五是加快新成果新装备推广应用。}2019年2月,150多位专家深入黑龙江省大豆播种面积超过10万亩的45个县(市、区),由以往的义务授课变成课题制,责任到人,技术到户,将45个县的培训落实成45个课题,每个县由1到2个培训小分队进行培训,提高了农民参与培训的积极性,有效实现了培训的实时化、在线化和长效化。

\textbf{六是开展大豆绿色高质高效行动。}在东北、黄淮海、西南地区,选择大豆面积具有一定规模、产业基础较好的县,开展整建制大豆绿色高质高效行动,示范推广高产优质大豆新品种。力争在东北、黄淮海地区分别创建一批亩产超150公斤、200公斤的大豆高产示范县,在西南地区创建一批玉米亩产超500公斤、大豆亩产超100公斤的“双高产”示范县。

\textbf{二、花生产业}

\textbf{一是技术推广服务持续深化,花生产业标准化产业化程度不断提高。}2013年,农业部油料专家指导组提出花生田间管理技术指导意见,此后每年给出春播和夏播花生生产技术指导意见;同时在全国各地以产业园形式建立花生生产加工产业链,促进花生产业链延伸;激发了花生种植积极性。

\textbf{二是生产试验区扩大带来的示范效应持续发挥,种植补贴范围不断扩大。}2017年以来,各地开始试行花生种植保险和补贴政策。作为配合精准扶贫战略的举措,花生规模种植作为解决精准扶贫中产业落地和防范返贫的措施被大量推广。

\section{2020—2021年供求形势判断}
\subsection{生产形势}

\textbf{大豆播种面积和产量持续增加。}随着大豆振兴计划的实施及大豆生产者补贴政策的落实,大豆种植面积和产量在一定时间内上升的可能性比较大。根据中国农业产业模型(CASM),2020年大豆播种面积增至949.84万公顷,同比增加1.77\%;单产增至1945.04千克/公顷,同比增加0.29\%;产量增至1847.47万吨,同比增加2.07\%。2021年大豆播种面积增至960.27万公顷,同比增加1.10\%;单产增至1965.23千克/公顷,同比增加1.04\%;产量增至1887.15万吨,同比增加2.15\%。

\textbf{油菜播种面积和产量继续下降。}2020年1月新冠肺炎疫情突发,给中国农业生产造成不同程度影响,油菜生产大省湖北、湖南和四川是此次疫情重灾区,虽然油菜正处越冬期,但后期油菜病虫害防控会受到一定影响。另外,油菜种植受天气影响较大,机械化程度低,种植收益缺乏优势,农民种植积极性依然不高。可见,中国油菜播种面积和产量增长乏力,2020—2021年播种面积和产量继续下降。根据中国农业产业模型(CASM),2020年油菜播种面积将降至648.48万公顷,2021年进一步降至644.84万公顷,分别较2019年减少2.83万公顷和6.47万公顷;2020—2021年油菜籽总产量分别降至1317.49万吨和1312.04万吨,较2018年水平分别减少4.47万吨和9.92万吨;油菜籽单产水平继续提升,但幅度较小,2020年和2021年单产水平分别达到2031.67千克/公顷和2034.69千克/公顷。

\textbf{花生播种面积小幅下降后稳步提升,单产和产量稳步提高。}由于当前影响花生市场的主要驱动因素为中美贸易摩擦,因此国内花生生产与外交情况密切相关。可以判断,至2020年下半年美国大选结束,中美贸易关系维持第一阶段协议的可能性较大。自美进口大豆会重新填补一部分因贸易摩擦产生的需求缺口,但其回弹程度有待观察,预计花生播种面积因此会出现小幅下降。另一方面,国内饲料需求又会催生一定的需求增加。根据中国农业产业模型(CASM),2020—2021年花生播种面积将分别达456.21万公顷和457.86万公顷。随着农业技术推广的继续扶持及技术下乡政策的稳步推进,预计2020年花生单产达3773.44千克/公顷,国内花生产量达1721.47万吨。

\subsection{消费形势}

\textbf{大豆工业需求推升总体消费。}随着人口增长和居民收入水平提高,大豆工业需求将稳步增加,大豆总消费量随之增加。根据中国农业产业模型(CASM),预计2020年大豆国内消费总量达到1.07亿吨,同比增加0.48\%;其中食用消费同比增加0.90\%,工业消费同比增加0.46\%。

\textbf{菜籽油消费需求稳步增长。}长江流域是中国传统油菜籽主产区,也是菜籽油主要消费区,包括长江上游的四川、重庆、贵州,中游的湖北、湖南、安徽,下游的江苏、上海和浙江等省市。受新冠肺炎疫情影响,餐饮行业消费需求将明显减少,但居民户内需求有所增加,将抵消户外消费量减少。随着中国经济发展水平提高,特别是城镇化进程不断向前推进,包括菜籽油在内的植物性油脂需求将保持刚性增长,菜籽油消费总体上也将继续保持增长。根据中国农业产业模型(CASM),2020年和2021年,油菜籽加工(压榨)需求达到1756.68万吨和1765.71万吨,分别比2019年增加9.43万吨和18.47万吨;菜籽油食用消费需求达到845.39万吨和849.88万吨,分别比2019年增加5.84万吨和10.33万吨。

\textbf{花生需求将出现小幅下降态势。}预计2020年,伴随中美贸易谈判第一阶段协议的实施,大豆在中国油料作物中的地位将复归,用于压榨食用油的花生需求可能受到一定程度的限制,但作为饲料用途的需求会在一定程度上缓解工业消费方面受到的影响,食用花生需求则在目前水平上进一步增加。根据中国农业产业模型(CASM),2020年花生需求总量为1722.83万吨。其中,食用消费714.72万吨,生产消费1008.11万吨。

\subsection{贸易形势}

\textbf{大豆净进口呈增长态势。}中国需要进口大豆以满足生猪产能恢复及疫情控制后养殖业恢复的豆粕需求。目前,中美已签署第一阶段贸易协议,中国有继续增加美豆购买的趋势。根据中国农业产业模型(CASM),2020年大豆净进口将增至8853.46万吨,2021年为8818.33万吨。

\textbf{油菜籽和菜籽油净进口继续保持双增长。}由于国内油菜籽产量下降,而菜籽油需求保持稳步增长,出现供需缺口,需依赖一定进口才能满足国内多样化需求。根据中国农业产业模型(CASM),2020年油菜籽净进口增至379.88万吨,较2019年增加15.13万吨;菜籽油净进口增至163.76万吨,较2019年提高2.76万吨。2021年油菜籽净进口增至397.57万吨,较2019年增加32.81万吨;菜籽油净进增至165.34万吨,较2019年增加4.34万吨。

\textbf{花生维持净进口,花生油净进口扩大,花生粕净出口下降。}由于国内花生供给充足,花生与花生油进口需求较小,考虑到2019年国内花生生产增速较快,在2020年油料作物需求缺口可能被自美进口大豆填补的预期下,国内对于花生和花生油的进口需求将进一步收缩。根据中国农业产业模型(CASM),2020年花生净进口量将达1.36万吨,花生油净进口量将达20.43万吨,花生粕净出口量为62.93万吨。

\begin{table}[]
\centering
\begin{threeparttable}
\textbf{\caption{2020—2021年中国主要油料作物生产预测}}
\setlength{\tabcolsep}{7mm}
\small
\begin{tabular}{lrrrrrr}
\hline
\multirow{2}{*}{品种} & \multicolumn{2}{c}{播种面积(万公顷)} & \multicolumn{2}{c}{单产(千克/公顷)} & \multicolumn{2}{c}{总产量(万吨)}    \\
                    & 2020年         & 2021年         & 2020年         & 2021年         & 2020年        & 2021年        \\
\hline
大豆                  & 949.84        & 960.27        & 1,945.04      & 1,965.23      & 1,847.47     & 1,887.15     \\
油菜籽                 & 648.48        & 644.84        & 2,031.67      & 2,034.69      & 1,317.49     & 1,312.04     \\
花生                  & 456.21        & 457.86        & 3,773.44      & 3,797.19      & 1,721.47     & 1,738.56     \\
合计                  & 2,054.52      & 2,062.96      & 2,378.38      & 2,393.53      & 4,886.43     & 4,937.75      \\
\hline
\end{tabular}
\begin{tablenotes}
\item \tiny{资料来源:中国农业产业模型(CASM)。}
\end{tablenotes}
\end{threeparttable}
\end{table}

\begin{table}[]
\centering
\begin{threeparttable}
\textbf{\caption{2020—2021年中国主要油料作物需求和贸易预测}}
\setlength{\tabcolsep}{7mm}
\small
\begin{tabular}{lrrrrrr}
\hline
\multirow{2}{*}{品种} & \multicolumn{2}{c}{食用需求(万吨)} & \multicolumn{2}{c}{工业需求(万吨)} & \multicolumn{2}{c}{净进口量(万吨)} \\
                    & 2020年         & 2021年         & 2020年         & 2021年         & 2020年        & 2021年        \\
\hline
大豆                  & 719.75        & 722.40       & 9,485.06      & 9,491.77     & 8,853.46      & 8,818.33     \\
油菜籽                 & /             & /            & 1,756.68      & 1,765.71     & 379.88        & 397.57       \\
花生                  & 714.72        & 727.31       & 1,008.11      & 1,009.07     & 1.36          & -2.19        \\
合计                  & 1,434.47      & 1,449.71     & 6,437.73      & 6,355.56     & 9,234.70      & 9,213.72     \\
\hline
\end{tabular}
\begin{tablenotes}
\item \tiny{资料来源:中国农业产业模型(CASM)。}
\end{tablenotes}
\end{threeparttable}
\end{table}

\section{国际竞争力分析}
\textbf{大豆国际市场竞争力弱。}从国际市场占有率看,近年全球大豆产量最大的六个国家中,巴西竞争力最强,美国其次,阿根廷也具有较强优势。中国竞争力较其他几个大豆主产国有明显劣势,国际市场占有率从2014年的0.34\%降至2018年的0.17\%。巴西近五年国际市场占有率均值达43.57\%,2018年升至56.08\%;美国国际市场占有率虽总体呈下降趋势,但竞争力仍然显著,近五年国际市场占有率均值达到37.47\%。就贸易竞争力指数而言,巴拉圭最强,巴西其次,美国贸易竞争力指数位列第三。中国竞争力较低,近五年贸易竞争力指数均值为-0.993。从显示性比较优势指数看,巴拉圭最强,巴西其次,美国位列第三。中国竞争力较低,近五年显示性比较优势指数均值为0.049。巴拉圭和巴西显示性比较优势均较明显,巴拉圭显示性比较优势指数由2014年的10.62升至2018年的11.61,巴西则由7.84升至10.85。美国显示性比较优势指数近五年在波动中略微下降,由2014年的3.90降到2018年的3.06。

\textbf{油菜产业不具备国际竞争优势。}国际市场上油菜籽主要生产国家和地区为加拿大、欧盟、俄罗斯、乌克兰和澳大利亚,其中加拿大是世界最大油菜籽生产和出口国,国际市场占有率高达42.9\%,贸易竞争力指数达0.943,显示性比较优势指数达11.22,三项指标均居世界首位。欧盟国际市场占有率为32.67\%,位列世界第二,但贸易竞争力指数和显示性比较优势指数较低。乌克兰国际市场占有率约为9.8\%,仅次于加拿大和欧盟,贸易竞争力指数达0.938,居世界第二位;显示性比较优势指数为9.054,处于世界第三位置。澳大利亚国际市场占有率为9.3\%,与乌克兰接近;显示性比较优势指数为9.271,仅次于加拿大;但澳大利亚贸易竞争力指数较低,仅为0.003。俄罗斯国际市场占有率为1.8\%,明显低于上述国家;贸易竞争力指数为0.626,高于欧盟和澳大利亚,位列世界第三;显示性比较优势指数相对较低,为0.947,略高于欧盟。中国油菜籽及其制品出口量较小,在国际市场上为净进口国,且净进口数量在不断增加;各国际竞争力评价指标显示,中国油菜产业国际竞争力较弱,不具备比较优势。

\textbf{花生产业处于竞争劣势。}2000—2018年,中国花生国际市场占有率在初期有所上升,中后期大幅下降;2003年国际市场占有率达到峰值,为35.78\%,随后持续降低,此后经过一定程度恢复,2018年也仅为11.84\%。贸易竞争力指数在维持相当长时间的较高水平后,2009年开始出现下降,2018年有所恢复,维持在0.412的水平。显示性比较优势指数方面,2003年起开始出现比较优势衰退,但至2010年仍为3.295,比较优势较强;此后继续下降,2016年降至极值1.386,而后有所回升。但是,在整体市场内部化的大趋势下,中国花生很难恢复到21世纪初的高比较优势状态。总体来说,2000年以来中国花生在国际竞争中逐渐丧失优势,与主要花生出口国相比,中国花生已处劣势。放眼国际,印度花生尽管近年来遭遇一些衰退,但始终保持较强竞争力和比较优势;缅甸花生产业后来居上,近年发展势头有赶超印度趋势;尼日利亚花生比较优势相对较弱,但在持续发展;苏丹和坦桑尼亚花生则出现明显竞争衰退,正逐步失去国际市场。

\begin{table}[]
\centering
\begin{threeparttable}
\textbf{\caption{1995—2018年中国和主要国家大豆贸易竞争力指数和显示性比较优势指数}}
\setlength{\tabcolsep}{6mm}
\small
\begin{tabular}{lcccccc}
\hline
年份        & 中国     & 巴西     & 阿根廷   & 美国    & 印度     & 巴拉圭    \\
\hline
贸易竞争力指数   &        &        &       &       &        &        \\
1995      & 0.138  & 0.587  & 0.999 & 0.988 & -0.993 & 0.999  \\
2005      & -0.957 & 0.975  & 0.872 & 0.977 & 0.999  & 0.934  \\
2010      & -0.991 & 0.993  & 1.000 & 0.977 & 0.998  & 0.989  \\
2016      & -0.994 & 0.988  & 0.819 & 0.986 & 0.494  & 0.996  \\
2017      & -0.995 & 0.993  & 0.592 & 0.977 & 0.626  & 0.992  \\
2018      & -0.995 & 0.996  & /     & 0.977 & 0.219  & 0.996  \\
\hline
显示性比较优势指数 &        &        &       &       &        &        \\
1995      & 0.530  & 3.918  & 3.764 & 5.373 & 0.000  & 20.815 \\
2005      & 0.317  & 8.188  & 6.425 & 4.095 & 0.010  & 21.605 \\
2010      & 0.073  & 6.037  & 4.588 & 4.350 & 0.011  & 12.618 \\
2016      & 0.043  & 7.565  & 2.631 & 4.449 & 0.089  & 10.067 \\
2017      & 0.035  & 8.742  & 2.293 & 4.095 & 0.128  & 11.598 \\
2018      & 0.037  & 10.853 & /     & 4.350 & 0.096  & 11.612    \\
\hline
\end{tabular}
\begin{tablenotes}
\item \tiny{资料来源:基于FAOSTAT、UN Comtrade和WTO数据计算得到。}
\item \tiny{注:“/”代表缺少数据。}
\end{tablenotes}
\end{threeparttable}
\end{table}

\begin{table}[]
\centering
\begin{threeparttable}
\textbf{\caption{1995—2018年中国和主要国家(或地区)油菜籽贸易竞争力指数和显示性比较优势指数}}
\setlength{\tabcolsep}{6mm}
\small
\begin{tabular}{lcccccc}
\hline
年份        & 中国     & 加拿大    & 欧盟     & 乌克兰   & 澳大利亚   & 俄罗斯    \\
\hline
贸易竞争力指数   &        &        &        &       &        &        \\
1995      & -0.963 & 0.971  & -0.158 & 0.622 & 0.967  & -1.000 \\
2005      & -0.999 & 0.946  & 0.030  & 0.827 & 0.998  & -0.799 \\
2010      & -1.000 & 0.936  & -0.153 & 0.908 & 0.997  & -0.637 \\
2016      & -0.999 & 0.952  & -0.204 & 0.973 & 0.996  & 0.331  \\
2017      & -1.000 & 0.963  & -0.229 & 0.827 & 0.997  & -0.447 \\
2018      & -1.000 & 0.943  & /      & 0.908 & 0.993  & 0.626  \\
\hline
显示性比较优势指数 &        &        &        &       &        &        \\
1995      & 0.011  & 11.165 & 0.876  & 0.145 & 2.046  & 0.000  \\
2005      & 0.001  & 9.377  & 0.873  & 3.368 & 8.106  & 0.056  \\
2010      & 0.000  & 11.451 & 1.017  & 8.297 & 5.848  & 0.065  \\
2016      & 0.002  & 11.612 & 0.969  & 2.415 & 10.231 & 0.333  \\
2017      & 0.000  & 11.509 & 0.895  & 3.368 & 12.177 & 0.153  \\
2018      & 0.000  & 11.216 & /      & 8.297 & 9.271  & 0.947    \\
\hline
\end{tabular}
\begin{tablenotes}
\item \tiny{资料来源:基于FAOSTAT、UN Comtrade和WTO数据计算得到。}
\item \tiny{注:“/”代表缺少数据。}
\end{tablenotes}
\end{threeparttable}
\end{table}

\begin{table}[]
\centering
\begin{threeparttable}
\textbf{\caption{1995—2018年中国和主要国家花生贸易竞争力指数和显示性比较优势指数}}
\setlength{\tabcolsep}{6mm}
\small
\begin{tabular}{lcccccc}
\hline
年份        & 中国     & 印度     & 缅甸     & 尼日利亚   & 苏丹     & 坦桑尼亚   \\
\hline
贸易竞争力指数   &        &        &        &        &        &        \\
1995      & 0.997  & 1.000  & 1.000  & 1.000  & 1.000  & 1.000  \\
2005      & 1.000  & 1.000  & -0.167 & -0.988 & 1.000  & 0.978  \\
2010      & 0.842  & 0.999  & -1.000 & -0.991 & -0.963 & -0.699 \\
2016      & -0.064 & 1.000  & 0.740  & -1.000 & -0.885 & -0.927 \\
2017      & 0.166  & 0.994  & 0.978  & -0.988 & -1.000 & -0.849 \\
2018      & 0.412  & 0.994  & /      & -0.991 & 0.000  & 0.000  \\
\hline
显示性比较优势指数 &        &        &        &        &        &        \\
1995      & 9.993  & 7.000  & 0.175  & 0.466  & 3.579  & 0.095  \\
2005      & 9.583  & 8.260  & 0.012  & 0.860  & 3.438  & 2.438  \\
2010      & 3.295  & 16.740 & 0.000  & 0.007  & 0.184  & 0.959  \\
2016      & 1.386  & 14.859 & 5.669  & 0.000  & 0.069  & 0.033  \\
2017      & 1.784  & 12.468 & 11.635 & 0.860  & 0.000  & 0.264  \\
2018      & 2.583  & 9.005  & /      & 0.007  & 0.000  & 0.000    \\
\hline
\end{tabular}
\begin{tablenotes}
\item \tiny{资料来源:基于FAOSTAT、UN Comtrade和WTO数据计算得到。}
\item \tiny{注:“/”代表缺少数据。}
\end{tablenotes}
\end{threeparttable}
\end{table}

\chapter{其他重要农产品产业}
\begin{titledbox}{\pfhei{主要观点}}
\begin{itemize}
  \item 2019年,马铃薯、蔬菜、水果生产保持增长,棉花种植面积及产量稳中有降,糖料作物出现“减种增产”;马铃薯消费总量增长但饲料消费比重下滑,棉花加工需求下降,食糖消费相对平稳,蔬菜、水果消费需求强劲;马铃薯贸易呈净出口态势,棉花、水果进口扩大,食糖出口劣势得到逆转,白砂糖进出口“双双疲软”,蔬菜出口形势向好。
  \item 2020—2021年,马铃薯、棉花、蔬菜、生产保持增长,食糖产量呈微幅下降,水果生产小幅下降或持平;马铃薯、食糖、蔬菜、水果消费继续增长,棉花消费呈下滑态势;马铃薯、蔬菜继续保持净出口态势,棉花进口量下降,食糖净进口大幅增长,水果逆差格局短期难以转变。
蔬菜、水果消费需求强劲;马铃薯贸易呈净出口态势,棉花、水果进口扩大,食糖出口劣势得到逆转,白砂糖进出口“双双疲软”,蔬菜出口形势向好。
  \item 马铃薯、蔬菜具备较强竞争优势,棉花国际市场竞争力弱;食糖国际竞争力大幅提升,出口劣势转为优势;近年水果国际竞争优势下行。
 \end{itemize}
\end{titledbox}
本章主要针对马铃薯、棉花、糖、蔬菜、水果等重要农产品的2019年产业发展新动态、2020—2021年供求形势及近年国际竞争力进行判断。

\section{2019年产业发展新动态}
\subsection{生产形势}
\textbf{马铃薯产量继续保持稳定增长态势,单产水平逐步提高。}2018年,中国马铃薯收获面积和总产量分别为533万公顷和10071万吨,排世界第一。根据中国农业产业模型(CASM),2019年马铃薯播种面积同比增长1.02\%,至538.33万公顷;产量同比增长1.99\%,至10271.29万吨,继续稳居世界首位。中国马铃薯单产水平2018年为18.90吨/公顷,2019年微增0.96\%,达到19.08吨/公顷。

\textbf{自然灾害对棉花单产产生影响,种植面积和产量稳中略降。}2019年,全国棉花种植面积为334.00万公顷,比2018年减少2.70万公顷,降幅为0.80\%。由于棉花生长关键期内遭受风沙、高温、冰雹、低温冻害等自然灾害,新疆、河北、江西、山东、湖北等主要产棉区的棉花单产均有不同程度下降。2019年,全国棉花总产量589.00万吨,比2018年减少15.00万吨,降幅为2.48\%;棉花单位面积产量1763.47千克/公顷,比2018年减少30.41千克/公顷,降幅为1.70\%。

\textbf{糖料作物出现“减种增产”现象,食糖总产增加。}2019年,中国糖料播种面积为162.00万公顷,同比微降1.81‰;但糖料产量达到12204.00万吨,同比增长2.33\%。可见,糖料单产在科技进步等因素的推动下有所提升。食糖种类较多,但主要分为白砂糖、红糖和冰糖三种。2019年,中国食糖产量为1076.00万吨,同比增长4.47\%。

\textbf{蔬菜生产保持增长态势。}2018年,中国蔬菜生产继续保持增长态势,全年产量达到70346.72万吨,比2017年下降13.30\%;蔬菜播种面积为2043.89万公顷,平均单产约34417.99千克/公顷。从世界范围看,中国蔬菜面积和产量仍居世界首位,保持蔬菜生产大国地位。从生产区域变化趋势看,中国蔬菜生产进一步向优势生产区集中,根据季节变化,全国六大产区产量稳步提高。随着设施蔬菜技术推广及面积扩大,蔬菜生产标准化、规模化、设施化及综合生产能力得到有效提升,保障了城乡居民蔬菜消费需求。根据农业农村部生产监测数据,2019年蔬菜生产呈稳定态势,蔬菜供需矛盾不突出,常年总量平衡有余,且品种丰富,蔬菜价格总体相对平稳,利于蔬菜种植农户调整市场预期。

\textbf{水果生产持续增长,供给总体稳定。}2018年,中国水果总产量为2.57亿吨,较2017年小幅增加。2019年水果种植面积较2018年略有增加,主产区气象条件基本适宜,苹果、梨、柑橘等大宗水果品种丰产,为全年水果供给充足奠定了基础。2018年,苹果、柑橘、梨占水果总产量的比重分别为15.3\%、16.1\%、6.3\%,其他小宗品种如桃、西瓜、甜瓜、红枣、葡萄、火龙果产量在需求多样化的刺激下产量占比逐步稳定。由于水果产业成本利润率普遍高于大宗农产品,许多贫困地区将水果作为脱贫攻坚的重要产业。近10年,水果种植面积总体呈稳步增长态势,柑橘、葡萄种植面积快速增加,2018年达244万公顷;苹果、香蕉面积分别保持在195万公顷和35万公顷水平;梨种植面积小幅下降。水果供给整体充足,生产者价格弱势运行。

\subsection{消费形势}
\textbf{马铃薯消费1亿吨以上,饲料消费比重有所下降。}根据中国农业产业模型(CASM),2019年马铃薯消费总量继续扩大,年消费量达到1.03亿吨,同比增长1.99\%。其中,鲜食马铃薯仍是最主要的消费形式,达到0.63亿吨左右。受非洲猪瘟疫情影响,生猪饲养量下降,马铃薯饲料消费比重有所下降,占总消费的比重从2018年的16\%降至2019年的13\%。

\textbf{棉花加工需求下降。}在宏观经济乏力和中美贸易摩擦的背景下,棉花下游市场需求走弱,叠加外纱进口的挑战。根据中国农业产业模型(CASM),2019年中国棉花消费量减少至768.80万吨,相比2018年下降33.0万吨,降幅为4.14\%。

\textbf{食糖消费相对平稳,库存下降明显。}根据中国农业产业模型(CASM),2019年,中国食糖总消费量为1466.00万吨,同比微增1.14\%;人均食糖食用消费量为11.29千克,同比微增0.3\%。就食糖用途而言,食糖食用消费1580.00万吨,同比增长6.37\%;食糖库存下降114.00万吨,同比增长8.36\%。

\textbf{蔬菜在外消费扩大,消费需求依旧强劲。}根据国家统计局数据,2018年,城镇居民蔬菜及食用菌人均消费量达到103.1千克,同比下降3.37\%;鲜菜人均消费量达到99.0千克,同比下降3.41\%。农村居民蔬菜及食用菌人均消费量达到87.5千克,同比下降2.99\%;鲜菜人均消费量达到85.6千克,同比下降3.28\%。虽然2018年城乡蔬菜人均消费量较2017年略有下降,但居民在外饮食消费扩大,实际蔬菜消费需求仍较为强劲。根据《中国居民膳食指南(2016版)》,成年人蔬菜日均摄入量应在300~500克之间,折算成全年摄入量应在109.45~182.50千克。因此,随着居民健康认知的提高及膳食平衡的需要,城乡居民蔬菜消费仍然存在进一步上升空间。

\textbf{水果消费呈快速增长态势。}根据中国农业产业模型(CASM),2018年城镇年人均消费水果量已达94千克,农村为67千克。相对于其他大宗农产品,水果消费易受多方因素综合影响。对外贸易环境优化和物流改善,生鲜农产品电子商务、线上平台和水果连锁店的快速发展,蓝莓、车厘子、牛油果、荔枝、火龙果等热带水果和进口水果已进入平常百姓家。水果地域性消费特征逐渐消失,许多产量不大、但又有明显地域和时令特色的水果供给质量和数量均有所提高,水果消费结构向多样化、优质化、品牌化转变。

\subsection{贸易形势}
\textbf{马铃薯保持净出口态势。}根据海关数据统计,2019年1—11月中国马铃薯及其制品总出口量已达到49万吨,其中鲜或冷藏马铃薯(种用除外)出口量达到45万吨。根据中国农业产业模型(CASM),2019年中国马铃薯净出口量达36.00万吨,继续保持净出口态势。

\textbf{棉花进口量扩大。}中国棉花产销缺口需要进口棉予以补充。海关统计数据显示,2019年中国累计进口棉花184.60万吨,净进口179.80万吨。贸易方式以一般贸易为主,占比33\%。受进口美棉加征关税和巴西连年丰产影响,2019年中国棉花进口结构较往年有明显调整,美棉进口比例显著下降至近20年以来的最低值,巴西成为第一棉花进口来源国,占总进口量的27\%。

\textbf{食糖净进口出现下滑,白砂糖进出口“双双疲软”。}中国是世界第二大食糖进口国,2019年食糖自给率仅为73.40\%,整体偏低,净进口数量相对较大。当年食糖净进口量为390.00万吨,与2018年的415.6万吨相比下降了6.16\%。白砂糖长期处于供不应求状态,每年均需进口弥补缺口。根据布瑞克农业数据终端,2019年,白砂糖净进口量为285万吨,同比下降6.50\%。其中,白砂糖进口量为300万吨,同比下降7.41\%;出口量仅15万吨,同比下降21.88\%。

\textbf{蔬菜贸易形势良好。}2018年,中国蔬菜贸易保持良好发展态势,与2017年相比,进出口额均呈现增长趋势。根据海关数据统计,2018年蔬菜出口额为152.4亿美元,进口额为8.3亿美元,与2017年相比,进出口额分别增长了1.8\%和50.0\%。海关数据显示,2019年1—10月,中国蔬菜出口额为124.4亿美元,同比增长0.2\%;进口额达到7.9亿美元,同比增长19.9\%;贸易顺差116.5亿美元,同比下降0.9\%。

\textbf{水果贸易呈净进口态势。}根据海关数据统计,2019年,中国水果进口量为729.3万吨,进口额为103.6亿美元,同比分别增长23.1\%和23.2\%;出口量为492.1万吨,同比减少3.5\%,出口额为74.5亿美元,同比增加4.1\%。贸易逆差为29.1亿美元,较2018年的12.59亿美元增长了131.1\%。

\section{2020—2021年供求形势判断}
\subsection{生产形势}
\textbf{马铃薯产量继续稳定增长。}马铃薯在我国大多地区都较适宜,有长期种植经验,产量大、种植面积广。2020—2021年,中国马铃薯总产量仍将保持世界领先地位。根据中国农业产业模型(CASM),2021年马铃薯总产量将突破1.07亿吨,单产水平稳步提高,有望达到20000千克/公顷。

\textbf{国内棉花生产仍以稳定为主。}受劳动力及土地成本持续上升、植棉比较效益降低、产业链下游消费低迷等因素影响,2020—2021年中国棉花种植面积将呈稳中略降态势。受单产提高支撑,棉花总产量较2019年略有增长。根据中国农业产业模型(CASM),2020年棉花播种面积为333.85万公顷,产量达592.70万吨;2021年棉花播种面积为333.35万公顷,产量达595.86万吨。

\textbf{食糖产量呈微幅下降态势。}受新冠肺炎疫情等诸因素综合影响,2020年食糖产量预计呈下降态势。据中国农业产业模型(CASM),2020年中国食糖产量为1065.74万吨,同比下降0.95‰;2021年食糖产量为1063.72万吨,同比下降0.19‰。

\textbf{蔬菜生产形势继续向好。}从2019年蔬菜生产情况看,中国蔬菜全年平稳运行,供给充足,为2020年提供了良好的产业发展基础;气候条件利于北方设施蔬菜生产,2019年冬季蔬菜供给充足,价格平稳。2020年,春季蔬菜供需持续平稳,除少数小品种产量波动较大外,大部分蔬菜总体供需宽松。2020年1—2月新冠肺炎疫情对蔬菜产业影响主要体现在短期,长期看,蔬菜产业将保持平稳发展态势。根据中国农业产业模型(CASM),预计2020年,蔬菜播种面积略增至2034.29万公顷,2021年降为2022.30万公顷。单产增长推动总产量呈上升态势,2020年蔬菜产量将达70994.33万吨,比2019年增长0.45\%;2021年达71098.80万吨,比2020年增长0.15\%。

\textbf{水果生产小幅下降或持平。}2020年一季度新冠肺炎疫情造成海南、广西等南方柑橘类水果和热带水果滞销,受疫情影响较大的湖北、重庆等南方柑橘主产区出现种苗运不出、成熟果滞留树上的现象。同时,疫情还导致苹果、梨等贮藏果出库量不畅。预计2020年,中国水果产量将小幅下降或持平。

\subsection{消费形势}
\textbf{马铃薯加工需求呈快速增长趋势。}随着国内马铃薯主食化战略推进,根据中国农业产业模型(CASM),2020年马铃薯加工需求可望比2019年增长7.51\%,达到1448.20万吨;2021年将继续增长至1553.55万吨。在加工需求带动及生猪恢复性增长影响下,2020年马铃薯总消费量将达到1.05亿吨,比2019年增长2.29\%。

\textbf{棉花消费呈现下滑态势。}在世界经济增速放缓、贸易保护主义抬头双重压力下,中国纺织品服装出口受阻局面短期难以突破;东南亚新兴国家以低廉人工成本、低关税优惠等优势加速竞争世界纺织产业布局,竞争优势凸显;化纤制造业将进一步以低成本和科技创新优势挤占原棉市场;全球棉花供求未来仍呈整体宽松趋势,外棉和外纱阶段性价格优势明显。可见,未来中国棉花消费将呈下降态势,但由于纺织产能基础和产品质量不断提升,中国作为全球最大棉花消费国的地位不会改变。根据中国农业产业模型(CASM),2020年棉花总消费量为756.14万吨,较2019年减少12.66万吨;2021年棉花总消费量预计与2020年基本持平。

\textbf{食糖消费呈平稳增长趋势。}根据中国农业产业模型(CASM),2020年中国食糖总消费1509.01万吨,同比增长2.93\%;食糖食用消费1610.63万吨,同比增长1.94\%;人均食糖食用消费11.46千克,同比增长1.55\%。2021年,食糖总消费1588.41万吨,同比增长5.26\%;食糖食用消费1639.13万吨,同比增长1.77\%;人均食糖食用消费11.63千克,同比增长1.45\%。

\textbf{蔬菜加工需求所占比重呈上升趋势。}受现代居民生活节奏和消费习惯改变等因素影响,蔬菜加工需求所占比重将呈上升趋势。根据中国农业产业模型(CASM),2020年蔬菜加工需求占比达到17.34\%。2020年,蔬菜短期内价格有略上升趋势,对部分地区菜农增收带来利好影响。从城镇化发展趋势、农产品电子商务、冷链运输体系完善等因素看,蔬菜产业未来发展前景更加趋于良好,消费端对便利化、安全化、营养化、有机化蔬菜产品需求旺盛,对多样化开发的蔬菜加工制品消费需求也稳定增长。在乡村振兴战略实施背景下,蔬菜已经成为促进农民收入提高和贫困地区脱贫致富的重要产业,应继续完善蔬菜产业规划,鼓励支持小农户参与蔬菜产业链增值环节,不断提高自身素质,掌握更加先进的蔬菜种植技术和管理模式,同时加强蔬菜产业科技创新,逐步推广智能化、信息化生产,精准适应市场消费变化,扩大品牌蔬菜、有机蔬菜等优质蔬菜产品供给,保证国民健康营养。

\textbf{水果消费保持增长态势。}全国水果食用消费量在供给充足推动下,将继续保持增长。根据中国农业产业模型(CASM),2020年水果水用消费量将达到1.57亿吨。国外特色水果、热带水果以及与中国水果产季有时间差的水果,对国内果品生产形成强有力竞争。大众水果消费选择增加,不同品种水果受消费偏好、季节、水果品质、供需关系等影响,出现价格差异和波动,且该种现象已成常态。

\begin{table}[]
\centering
\begin{threeparttable}
\textbf{\caption{2018—2021年中国其他重要农产品供求和贸易}}
\setlength{\tabcolsep}{7mm}
\small
\begin{tabular}{llrrrr}
\hline
    &           & \multicolumn{1}{c}{2018年} & \multicolumn{1}{c}{2019年} & \multicolumn{1}{c}{2020年} & \multicolumn{1}{c}{2021年} \\
\hline
马铃薯 & 国内产量(万吨)  & 10,070.87 & 10,271.29 & 10,506.86 & 10,724.64 \\
    & 播种面积(万公顷) & 532.90    & 538.33    & 544.78    & 548.41    \\
    & 单产(千克/公顷) & 18,898.10 & 19,079.97 & 19,286.54 & 19,555.93 \\
    & 食用需求(万吨)  & 6,290.96  & 6,321.75  & 6,321.07  & 6,289.25  \\
    & 饲料需求(万吨)  & 1,618.86  & 1,374.20  & 1,276.75  & 1,471.70  \\
    & 工业需求(万吨)  & 953.66    & 1,347.06  & 1,448.20  & 1,553.55  \\
    & 净出口量(万吨)  & 34.38     & 36.00     & 245.87    & 176.01    \\
\hline
棉花  & 国内产量(万吨)  & 604.00    & 589.00    & 592.70    & 595.86    \\
    & 播种面积(万公顷) & 336.70    & 334.00    & 333.85    & 333.35    \\
    & 单产(千克/公顷) & 1,793.88  & 1,763.47  & 1,775.35  & 1,787.51  \\
    & 工业需求(万吨)  & 824.00    & 803.00    & 783.03    & 783.03    \\
    & 净进口量(万吨)  & 198.00    & 179.80    & 163.44    & 165.45    \\
\hline
糖   & 国内产量(万吨)  & 1,030.00  & 1,076.00  & 1,065.74  & 1,063.72  \\
    & 食用需求(万吨)  & 1,570.00   & 1,580.00   & 1,610.63   & 1,639.13     \\
    & 净进口量(万吨)  & 415.60    & 390.00    & 443.27    & 524.69    \\
\hline
蔬菜  & 国内产量(万吨)  & 70,346.72 & 70,673.53 & 70,994.33 & 71,098.80 \\
    & 播种面积(万公顷) & 2,043.89  & 2,039.45  & 2,034.29  & 2,022.30  \\
    & 单产(千克/公顷) & 34,417.99 & 34,653.31 & 34,898.83 & 35,157.44 \\
    & 食用需求(万吨)  & 24,034.72 & 24,333.21 & 24,487.01 & 24,596.88 \\
    & 工业需求(万吨)  & 12,027.00 & 12,169.00 & 12,308.52 & 12,439.72 \\
    & 净出口量(万吨)  & 899.00    & 596.68    & 741.58    & 1,058.34  \\
\hline
水果  & 国内产量(万吨)  & 25,688.35 & 25,827.11 & 25,983.95 & 26,115.34 \\
    & 播种面积(万公顷) & 1,187.49  & 1,185.80  & 1,184.76  & 1,183.51  \\
    & 单产(千克/公顷) & 21,632.53 & 21,780.29 & 21,931.83 & 22,065.99 \\
    & 食用需求(万吨)  & 15,419.25 & 15,581.37 & 15,711.33 & 15,818.59 \\
    & 工业需求(万吨)  & 5,137.67  & 5,142.07  & 5,145.78  & 5,148.67  \\
    & 净进口量(万吨)  & 6.24      & 348.00    & 330.51    & 298.46      \\
\hline
\end{tabular}
\begin{tablenotes}
\item \tiny{资料来源:中国农业产业模型(CASM)。}
\end{tablenotes}
\end{threeparttable}
\end{table}

\subsection{贸易形势}
\textbf{马铃薯及其制品将继续保持净出口态势。}中国马铃薯产量大,发展潜力也大,不但能够满足本国市场需求,而且每年都能有一定数量的出口,其中鲜或冷藏马铃薯(种用除外)是最具有竞争力的马铃薯产品,出口量稳定且逐年增加。根据中国农业产业模型(CASM),2020年中国马铃薯净出口量将达到新高。

\textbf{棉花进口量下降。}在不断去库存、产不足需形势下,进口棉仍是中国满足原棉消费需求的重要补充,特别是对“无三丝”、长绒棉等优质进口棉仍存在较大需求,未来原棉进口会增加中高端棉以补充国内消费缺口。受纺织产能转移和价格因素影响,进口纱线或挤占部分原棉进口。根据中国农业产业模型(CASM),2020—2021年中国棉花净进口量将分别达到163.44万吨和165.45万吨,相比2019年分别减少9.10\%和7.98\%。

\textbf{食糖净进口量出现大幅增长。}2019年,中国食糖净进口量降至390万吨,“拐点”初现端倪。根据中国农业产业模型(CASM),2020年中国食糖净进口量将反弹至443.27万吨,同比增长13.66\%;2021年净进口量达524.69万吨,同比增长18.37\%。

\textbf{蔬菜净出口保持增长态势。}中国蔬菜产业具有较强国际竞争力,综合考虑国内供需形势,预计蔬菜出口市场依旧保持向好态势。根据中国农业产业模型(CASM),2020年蔬菜净出口量将达到741.58万吨,同比增长24.28\%;2021年净出口量达到1058.34万吨,同比增长42.71\%。

\textbf{水果贸易逆差格局短期难以转变。}预计到2020年,水果贸易逆差格局难以逆转,进出口虽然由于第一季度新冠肺炎疫情受阻,但疫情结束后,水果进出口仍会保持恢复性增长,整体交易量将较2019年有小幅下降。在水果市场进一步开放的环境下,水果产业亟需发展社会化服务、生产托管、应用高效绿色等生产技术实现节本增效,各地也应着力开发适合当地自然和人力资源禀赋的水果品种,发展特色产业,采取差异化发展策略应对竞争挑战。

\section{国际竞争力分析}
\textbf{马铃薯具备较强出口比较优势。}中国马铃薯国际市场占有率从1995年的0.249\%升至2018年的6.343\%,市场份额不断扩大,始终保持强劲上扬势头。与马铃薯贸易强国德国、法国、荷兰等相比,中国马铃薯国际市场占有率增速虽较高,但国际市场份额相对较低。荷兰马铃薯国际市场占有率虽从1995年的27.141\%降至2018年的19\%,但仍保持世界领先地位。中国马铃薯贸易竞争力指数接近于1,具有较强净出口优势且净出口相对规模大。德国、比利时、埃及贸易竞争力指数在某些年份出现过负数,但已得到改观。中国马铃薯显示性比较优势指数从1995年的0.098提升到2018年的1.384,说明具有较强的出口比较优势。法国、荷兰相应指数多数年份大于2.5,具有很强的出口比较优势。

\textbf{棉花国际市场竞争力弱。}中国棉花国际市场占有率在2000年前后到达最高值4.7\%后,一直处于较低水平,2004年至今国际市场占有率在1\%以下。2018年全球棉花出口额为143.2亿美元,中国棉花出口份额仅占国际市场的0.65\%。美国棉花国际市场占有率最高,年均占有率为35.8\%,2018年达到45.7\%;印度是全球最大棉花生产国,棉花国际市场占有率从1995年的0.5\%升至2018年的15.4\%。除部分年份外,中国棉花贸易竞争力指数基本为负,说明出口竞争处于劣势,且净进口相对规模较大。主要产棉国印度、巴西贸易竞争力指数波动较大,但均大于0,说明两国棉花仍有较强竞争优势;美国、澳大利亚、乌兹别克斯坦贸易竞争力指数接近于1,说明三者具有很强出口竞争优势。1995年以来,中国棉花显示性比较优势指数仅在1999—2000年大于1,其余年份均小于1;2010年之后该指数下降明显,多个年份几乎接近于0,说明中国棉花出口竞争力弱,在全球棉花贸易市场上不具比较优势。

\textbf{食糖国际竞争力大幅提升,出口劣势转为优势。}2018年,中国食糖国际市场占有率同比提高4.32个百分点,达4.616\%,与澳大利亚相近,高于印度,低于巴西、泰国和欧盟,表明中国食糖国际竞争力总体较强且提升迅猛。同期,中国食糖贸易竞争力指数发生逆转,由负数(-0.845)转为正数(0.125),低于巴西、澳大利亚、泰国及印度,但高于欧盟,说明中国食糖2018年开始具有出口竞争优势。此外,中国食糖显示性比较优势指数从0.065增至1.007,同样低于澳大利亚、巴西、泰国及印度,但高于欧盟,表明2018年中国食糖出口从无比较优势开始具有一定比较优势。

\textbf{蔬菜产业国际竞争力强。}1995—2018年,中国蔬菜国际市场占有率保持波动上升趋势,表明蔬菜国际市场竞争力趋于增强。相比而言,日本、韩国、印度尼西亚、越南等的国际市场占有率均较低,不及1\%,缺乏竞争力。美国蔬菜国际市场占有率从1995年的9.375\%降到2018年的6.229\%,国际竞争力呈现一定下滑趋势。中国贸易竞争力指数保持波动略下降趋势,但基本都在0.93以上,表明中国蔬菜在国际市场上具有较强竞争力。日本贸易竞争力指数多数年份接近于-1,说明存在明显竞争劣势;韩国、美国、印度尼西亚等国家贸易竞争力指数均保持下降趋势,竞争力下降明显。中国蔬菜显示性比较优势指数一直大于2.5,较高年份甚至达到6.977,表明中国蔬菜竞争力很强,虽然近年有所下降,但总体仍处高位。韩国、美国蔬菜显示性比较优势指数近年不断下降,丧失出口比较优势;除极个别年份外,日本、印度尼西亚、越南等国家显示性比较优势指数均低于0.8,无出口比较优势。

\textbf{水果产业具备较强竞争优势,但近年形势有所下行。}2018年,水果进出口贸易都出现大幅增长,但罕见地出现了小幅逆差,逆差额为12.59亿美元。2019年,水果进口准入继续保持快速增长,有12个国家和地区的13种新鲜水果获得准入,新鲜水果准入的国家达到24个,品种超过200种。受中美贸易摩擦影响,2019年两国之间水果贸易量缩减。2019年,水果进出口贸易逆差进一步扩大至29.1亿美元。中国水果出口国际市场占有率从近10年10\%~12\%的水平下降了5个百分点。同时,根据贸易竞争力指数和显示性比较优势指数,2018年两者分别为0.376和1.174,中国水果进口仍具备一定竞争优势,但比较优势自2014年持续下滑。


\chapter{畜产品和水产品产业}
\begin{titledbox}{\pfhei{主要观点}}
\begin{itemize}
  \item xx。
 \end{itemize}
\end{titledbox}


\end{document}

%%% Local Variables:
%%% mode: latex
%%% TeX-master: t
%%% End:
